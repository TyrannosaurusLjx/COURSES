\documentclass[12pt, a4paper, oneside]{ctexart}
\usepackage{amsmath,extarrows, amsthm, amssymb, bm, graphicx, hyperref, geometry, mathrsfs,color}

\title{\huge\textbf{Lebesgue可测集与Lebesgue测度}}
\author{luojunxun}
\date{\today}
\linespread{2}%行间距
\geometry{left=2cm,right=2cm,top=2cm,bottom=2cm}%设置页面
\CTEXsetup[format={\Large\bfseries}]{section}%section左对齐

%定义环境
\newenvironment{Def}[1][def-name]{\par\noindent{\textit{(#1):}\small}}{\\\par}
\newenvironment{theorem}[1][Theorem-name]{\par\noindent \textbf{Theorem #1:}\textit}{\\\par}
\newenvironment{corollary}[1][corollary-name]{\par\noindent \textbf{Corollary #1:}\textit}{\\\par\vspace*{15pt}}
\newenvironment{lemma}[1][lemma-name]{\par\noindent \textbf{Lemma #1:}\textbf}{\\\par}
\renewenvironment{proof}{\par\noindent{\textit{Proof:}\small}}{\\\par}
\newenvironment{example}[1][example-name]{\par{\textbf{Example:}}}{\\\par}
\newenvironment{say}{\center{\textit{summary:}}}{\\\par}
\newenvironment{note}[1][note-name]{\par\textit{#1:}}{\\\par}


\begin{document}
\maketitle

\Def[外测度]{$m^*(E) = \inf\{\sum\limits_nl(I_n)|\{I_n\}_{n\geq 1}\text{是一列开区间并且}E\subset \bigcup\limits_nI_n\}$}

\Def[次可加性]{$m^*(\bigcup\limits_nE_n)\leq \sum\limits_nm^*(E_n)$}

\Def[可测集]{$E\subset R:if:m^*(E)\geq m^*(A\cap E)+m^*(A\cap E^c)$.就称E是勒贝格可测集,事实上这时候有$m^*(E)= m^*(A\cap E)+m^*(A\cap E^c)$;定义$m(E)=m^*(E)$}

用$\Omega$表示可测集全体

\Def[定理2.3.2]{外测度是零的集合是可测集,并且测度为零,称为零测集.换言之可数集是零测集}

区间的测度就是区间的长度\\





















\end{document}