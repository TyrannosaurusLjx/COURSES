\documentclass[12pt, a4paper, oneside]{ctexart}
\usepackage{amsmath,extarrows, amsthm, amssymb, bm, graphicx, hyperref, geometry, mathrsfs,color}

\title{\huge\textbf{PDE chapter2}}
\author{luojunxun}
\date{\today}
\linespread{2}%行间距
\geometry{left=2cm,right=2cm,top=2cm,bottom=2cm}%设置页面
\CTEXsetup[format={\Large\bfseries}]{section}%section左对齐

%定义环境
\newenvironment{Def}[1][def-name]{\par\noindent{\textit{(#1):}\small}}{\\\par}
\newenvironment{theorem}[1][Theorem-name]{\par\noindent \textbf{Theorem #1:}\textit}{\\\par}
\newenvironment{corollary}[1][corollary-name]{\par\noindent \textbf{Corollary #1:}\textit}{\\\par\vspace*{15pt}}
\newenvironment{lemma}[1][lemma-name]{\par\noindent \textbf{Lemma #1:}\textbf}{\\\par}
\renewenvironment{proof}{\par\noindent{\textit{Proof:}\small}}{\\\par}
\newenvironment{example}[1][example-name]{\par{\textbf{Example:}}}{\\\par}
\newenvironment{say}{\center{\textit{summary:}}}{\\\par}
\newenvironment{note}[1][note-name]{\par\textit{#1:}}{\\\par}
\newcommand{\qie}{\enspace\&\enspace}


\begin{document}
\maketitle

\section{$u_t+cu_x=0$}
解析方法:特征线方法:\par
特征线族:$\frac{dx}{dt}=c$,这样$x-ct=\xi$,解出$u(t,x)=f(x-ct)$,这里解描述了一个速度为c的右传波,表现为$u(0,x)=u(T,x+cT)=f(x)$.同理有左传波.

数值方法:有限差分:利用向后差分方法$\frac{v(t+k,x)-v(t,x)}{k}+c\frac{v(t,x)-v(t,x-h)}{h}=0\Rightarrow v(t+k,x) = [(1-\lambda c)+\lambda cE^{-1}]v(t,x)\Rightarrow v(t,x)=v(nk,x)=\sum_{i=0}^nC_n^i(1-\lambda c)^i(\lambda c)^{n-i}f(x-(n-i)h)  $,当$\lambda c\leq 1$时,符合CFL准则,并且这种情况下产生的误差和f的测量误差一样,从而稳定,可以证明这种差分格式收敛到经典解














% \begin{figure}[p]

%     \centerline{\includegraphics[width=1.2\linewidth,height=1.1\textheight]{name}}
%     \caption{课上习题}
%     \label{figure}

%\end{figure}



% \bibliographystyle{IEEEtran}
% \bibliography{reference}



\end{document}
