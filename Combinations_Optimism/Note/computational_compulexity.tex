\documentclass[12pt, a4paper, oneside]{ctexart}
\usepackage{amsmath,extarrows, amsthm, amssymb, bm, graphicx, hyperref, geometry, mathrsfs,color}

\title{\huge\textbf{计算复杂性}}
\author{luojunxun}
\date{\today}
\linespread{2}%行间距
\geometry{left=2cm,right=2cm,top=2cm,bottom=2cm}%设置页面
\CTEXsetup[format={\Large\bfseries}]{section}%section左对齐

%定义环境
\newenvironment{Def}[1][def-name]{\par\noindent{\textit{(#1):}\small}}{\\\par}
\newenvironment{theorem}[1][Theorem-name]{\par\noindent \textbf{Theorem #1:}\textit}{\\\par}
\newenvironment{corollary}[1][corollary-name]{\par\noindent \textbf{Corollary #1:}\textit}{\\\par\vspace*{15pt}}
\newenvironment{lemma}[1][lemma-name]{\par\noindent \textbf{Lemma #1:}\textbf}{\\\par}
\renewenvironment{proof}{\par\noindent{\textit{Proof:}\small}}{\\\par}
\newenvironment{example}[1][example-name]{\par{\textbf{Example:}}}{\\\par}
\newenvironment{say}{\center{\textit{summary:}}}{\\\par}
\newenvironment{note}[1][note-name]{\par\textit{#1:}}{\\\par}
\newcommand{\qie}{\enspace\&\enspace}

\newcommand{\mysum}[3]{$\sum\limits_{#1}^{#2}{#3}$}

\begin{document}
\maketitle

抽象的研究计算一个问题最少需要多少时间,而不考虑各种算法

优化问题和判定问题在求解上具有等价性,他们可以相互转化

描述一个实例所需的计算机存储单元数称为该规模的实例

实例的最大数是实例中出现过得最大整数,比如TSP问题的最大数$B=\max \{n,c_{i,j}\}$

不能被多项式控制的时间复杂度就成为指数时间算法,否则称为多项式时间算法

伪多项式时间算法,比如$O(nB),O(B\log n) $

多项式时间算法称为高效算法














% \begin{figure}[p]

%     \centerline{\includegraphics[width=1.2\linewidth,height=1.1\textheight]{name}}
%     \caption{课上习题}
%     \label{figure}

%\end{figure}



% \bibliographystyle{IEEEtran}
% \bibliography{reference}



\end{document}
