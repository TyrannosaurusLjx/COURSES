\documentclass{article}

\usepackage{ctex}
\usepackage{graphicx}
\usepackage{float}
\usepackage{datetime}
\usepackage{tikz}


\title{旋转群的有限子群}
\author{昨日旧友}
\date{\today}

\begin{document}
\maketitle

\part{}
这是抽象代数群的对称理论部分关于旋转群的有限子群的笔记,我本来想把这部分笔记像
往常一样打在幕布上的,但是我发现,我的英语水平不够我下次再看见我的笔记的时候
能反应过来,所以还是用中文叙述一下比较好\\


\part{}
    \section{所需知识}
        需要知道一些基本的群,比如$C_n$循环群之类的,
        然后,假定$G$是$SO_3$的有限子群,而且$G$的阶是$N$.$S$是空间中的单位球面。
        那么G中的每个元素g作用在$R^3$上表现为绕空间中过原点的一条直线$l$的旋转,
        这条直线和单位球有两个交点,我们称每个交点为一个$pole$(极点),用P表示所有极点构成的集合。
        再定义一个$pair:(g,p)$也就是极点和对应的稳定子构成的对,我们这里让g不等于恒等变换
        (因为恒等变换有无穷多个极点)
    \section{先证明$G$在$P$上作用}
        回顾$P$的定义,$P$中元素是那些经过$G$中某个元素$g$作用以后保持自身不变的元,要证明$G$在$P$上作用,
        就是说要去证明$P$中任意元经过$G$中一个给定元作用以后得到的生成元,还能被$G$中另外一个元保持不动.
        \\从而我们任取$g \in G , p \in P$并且$p$是$x\in G$的极点$i.e.xp=p$,则$gp$是生成元.\\
        那么$gxg^{-1} \in G$就是我们要找的.
        因为$(gxg^{-1})(gp)=gxp=gp$\\
        这就证明了$G$在$P$上作用

    \section{利用$pair$构造等式}
    通过固定$pair$中的$g$或者是$p$来对它的个数进行计数\\
    固定$g$:因为$G$的阶是$N$,除去恒等元,每个元有两个稳定子,这就是说一共有$2\times (N-1)$个$pair$\\
    固定$p$:$G$作用在每个$p\in P$上形成了一些不交的轨道,称这些轨道是$O_i,(i=1,2,..)$.
    \\(注意,平常用小写字母来标记每个轨道,但是这里我们并不清楚到底哪些元处在同一个轨道,所以我们用自然数做标记)
    \\那么我们把每个轨道上的元对应的$pair$全计算出来,再相加,就能得到总数这里我们给出一些记号\\
    $|O_p|=r_p,|G_p|=n_p$也就是用$r_p$和$n_p$分别表示$p$的稳定化子的阶和轨道的阶.
    再由计数公式$|G|=|O_p||G_p| \quad i.e.\;N=r_pn_p$,同一轨道的元的轨道的阶是一样的,从而稳定子的阶也是一样的,那么在每个轨道中,
    元素$p$对应$r_p-1$个非恒等的稳定子,那么就有$n_p(r_p-1)$个$pair$,对轨道个数求和就得到$\sum\limits_{i} n_i(r_i-1)$
    最终得到等式$$\sum\limits_{i} n_i(r_i-1)=2N-2$$
    两边同时除$N$得到$\sum\limits_{i} (1-\frac{1}{r_i})=2-\frac{2}{N}$这里$r_i\ge 2$因为至少有恒等元

    \section{数值分析}
    这样的话,左边的每个元大于等于$\frac{1}{2}$但是右边小于2,这就是说左边最多能求和三次,换句话说,最多有三条轨道!\\
    也就是说空间中的任何一个几何体在G的作用下形成的极点最多只能构成三个等价类.逐一考虑

        \subsection{一条轨道}
            不可能,因为$N$至少是2,这样的话右边至少是1,但是左边小于1

        \subsection{两条轨道}
            这样的话化成$\frac{N}{r_1}+\frac{N}{r_2}=2$但是$r_i|N$所以只能$r_1=r_2=N,n_1=n_2=1$
            \\也就是说$G$保持$p_1,p_2$不变,(这里我们反过来确定了$G$的种类),这样$G$是绕$p_1,p_2$连线直线的旋转群$C_n$
        \subsection{三条轨道}
        $$\frac{1}{r_1}+\frac{1}{r_2}+\frac{1}{r_3}-1=\frac{2}{N}$$
            这时候$r_1=2$,因为如果所有的$r_i$都至少为3,左边会小于零
            \subsubsection{$r_2=2$}
                这时$r_1=r_2=2,n_1=n_2=\frac{N}{2},r_3=\frac{N}{2},n_3=2$先来看轨道三:轨道三只有两个元,$G$中的元要么让他们保持不动,要么让他们互换.
                那么g要么是围绕以这两个极点为轴的的直线$l_{AB}$的旋转,要么是$l_{AB}$的翻转(即交换A和B).再结合轨道一和轨道二,认定$G$是一个让正$\frac{N}{2}$
                边形不动的二面体群$D_{\frac{N}{2}}$,这个正多面体就在轴AB的中垂面上,轨道一上的元可以认为是面的中点,轨道二上的元可以认为是正多面体的顶点
            \begin{tikzpicture}
                \draw [dashed][color=blue!50,->](0,0) node[left]{$A$}-- node [color=red!70,pos=0.25,above,sloped]{中心轴}(3,2) node[right]{$B$};
                \draw (1.5,1) ellipse (20pt and 10pt);   
                \fill (1.5,1) circle (2pt) node{$O$};
            \end{tikzpicture}

            "轨道一对应的是边,轨道二对应的是顶点,轨道三对应的是面"

            \subsubsection{$r_2 \ge 2$}
                如果$r_2=3$那么$r_3$最多是5,否则左边小于零,如果$r_3=4$,那么$r_3=3$(只能,但是这和我们默认使用的增序不符合).现在只有三种可能\\
                $(i):r_i=(2,3,3),n_i=(6,4,4),N=12$\\
                这是空间中保持正四面体不动的群$T$\\
                $(ii):r_i=(2,3,4),n_i=(12,8,6),N=24$\\
                这是空间中保持正八面体不动的群$O$\\
                $(iii):r_i=(2,3,5),n_i=(30,20,12),N=60$\\
                这是空间中保持正十二面体不动的群$I$\\

%如果你来看源文件,我想告诉你,我实在是写不下去了.作业还没做完啊

\end{document}
