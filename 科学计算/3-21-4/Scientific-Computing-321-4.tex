\documentclass[12pt, a4paper, oneside]{ctexart}
\usepackage{amsmath,extarrows, amsthm, amssymb, bm, graphicx, hyperref, geometry, mathrsfs,color}

\title{\huge\textbf{积分逼近}}
\author{luojunxun}
\date{\today}
\linespread{2}%行间距
\geometry{left=2cm,right=2cm,top=2cm,bottom=2cm}%设置页面
\CTEXsetup[format={\Large\bfseries}]{section}%section左对齐

%定义环境
\newenvironment{Def}[1][def-name]{\par\noindent{\textit{(#1):}\small}}{\\\par}
\newenvironment{theorem}[1][Theorem-name]{\par\noindent \textbf{Theorem #1:}\textit}{\\\par}
\newenvironment{corollary}[1][corollary-name]{\par\noindent \textbf{Corollary #1:}\textit}{\\\par\vspace*{15pt}}
\newenvironment{lemma}[1][lemma-name]{\par\noindent \textbf{Lemma #1:}\textbf}{\\\par}
\renewenvironment{proof}{\par\noindent{\textit{Proof:}\small}}{\\\par}
\newenvironment{example}[1][example-name]{\par{\textbf{Example:}}}{\\\par}
\newenvironment{say}{\center{\textit{summary:}}}{\\\par}
\newenvironment{note}[1][note-name]{\par\textit{#1:}}{\\\par}
\newcommand{\qie}{\quad\&\quad}


\begin{document}
\maketitle

\section*{梯形求积公式}:
$\int_a^b f(x)dx \approx \frac{b-a}{2}(f(a)+f(b))$

\section*{抛物线求积公式(Simpon公式)}
$\int_a^b f(x)dx \approx \frac{b-a}{6}(f(a)+4f(\frac{a+b}{2})+f(b))$

\section*{Newton-Cote公式:}
把 $[a, b]$ 区间 $n$ 等份, 其分点为
$$
x_i=a+i h, \quad i=0,1,2, \cdots, n, \quad h=\frac{b-a}{n} .
$$
过这 $n+1$ 个节点, 构造一个 $n$ 次多项式
$$
P_n(x)=\sum_{i=0}^n \frac{\omega(x)}{\left(x-x_i\right) \omega^{\prime}\left(x_i\right)} f\left(x_i\right)
$$
其中 $\omega(x)=\left(x-x_0\right)\left(x-x_1\right) \cdots\left(x-x_n\right)$. 用 $P_n(x)$ 代替 $f(x)$, 得
$$
\begin{aligned}
& \int_a^b f(x) d x \approx \int_a^b P_n(x) d x=\sum_{i=0}^n A_i f\left(x_i\right), \\
& A_i=\int_a^b \frac{\omega(x)}{\left(x-x_i\right) \omega^{\prime}\left(x_i\right)} d x . \\
&
\end{aligned}
$$



\section*{误差估计}
对于一个一般的估计公式:$\int_a^b f(x)dx\approx\sum\limits_{k=0}^nA_kf(x_k)$\\
其中$A_k$是不依赖于函数$f$的常数,如果对于任意的低于m次的多项式,等式精确成立,对任意m+1阶及以上多项式不能精确成立则称求积公式有m次代数精度

\note[note]{梯形求积公式的代数精度是1,Simpon公式的代数精度是3,Newton-Cote公式的代数精度是n,当n是偶数的时候,精度达到n+1}

\theorem[5.1]{ 若 $f(x) \in C^2[a, b]$, 则梯形求积公式有误差估计
$$
\begin{aligned}
R_T(f) & =\int_a^b f(x) d x-\frac{b-a}{2}[f(a)+f(b)] \\
& =-\frac{(b-a)^3}{12} f^{\prime \prime}(\eta), \quad a \leq \eta \leq b
\end{aligned}
$$}

\theorem[5.2]{ 若 $f(x) \in C^4[a, b]$, 则Simpson求积公式有误差估计
$$
\begin{aligned}
R_S(f) & =\int_a^b f(x) d x-\frac{b-a}{6}\left[f(a)+4 f\left(\frac{a+b}{2}\right)+f(b)\right] \\
& =-\frac{(b-a)^5}{2880} f^{(4)}(\eta), \quad a \leq \eta \leq b
\end{aligned}
$$}

\section*{复化公式及其误差估计}
$T_n$复化梯形公式:$\int_a^b f(x)dx=\sum\limits_{k=0}^{n-1}\int_{x_k}^{x_{k+1}}f(x)dx\approx\sum\limits_{k=0}^{n-1}\frac{x_{k+1}-x_k}{2}[f(x_k)+f(x_{k+1})]=
\frac{h}{2}(f(x_0)+f(x_1)+f(x_1)+\cdots+f(x_{n-1})+f(x_n))=\frac{h}{2}(f(a)+f(b)+2\sum\limits_{k=1}^{n-1}f(a+kh))$\\
若把区间 $2 n$ 等分, 在每个区间上仍用梯形求积公式, 则可 得 $T_n, T_{2 n}$ 之间的关系式
$$
\begin{gathered}
T_{2 n}=\frac{1}{2}\left(T_n+H_n\right) \\
H_n=h \sum_{k=1}^n f\left(a+(2 k-1) \frac{b-a}{2 n}\right) .
\end{gathered}
$$\vspace*{10pt}\\

因为Simpson公式用到区间的中点, 在构造复合Simpson公式时必 须把区间等分为偶数份. 为此, 令 $n=2 m, m$ 是正整数, 在每个 小区间 $\left[x_{2 k-2}, x_{2 k}\right]$ 上用Simpson求积公式
$$
\int_{x_{2 k-2}}^{x_{2 k}} f(x) d x \approx \frac{2 h}{6}\left[f\left(x_{2 k-2}\right)+4 f\left(x_{2 k-1}\right)+f\left(x_{2 k}\right)\right]
$$
其中 $h=\frac{b-a}{n}$, 因此
$$
\begin{aligned}
\int_a^b f(x) d x= & \sum_{k=1}^m \int_{x_{2 k-2}}^{x_{2 k}} f(x) d x \approx \sum_{k=1}^m \frac{h}{3}\left[f\left(x_{2 k-2}\right)+4 f\left(x_{2 k-1}\right)+f\left(x_{2 k}\right)\right] \\
& =\frac{h}{3}\left[f(a)+f(b)+4 \sum_{k=1}^m f\left(x_{2 k-1}\right)+2 \sum_{k=1}^{m-1} f\left(x_{2 k}\right)\right]=: S_n
\end{aligned}
$$
称 $S_n$ 为复合Simpson公式.\vspace*{10pt}\\

\theorem[5.3]{ 若 $f(x) \in C^2[a, b]$, 则
$$
R\left(f, T_n\right)=\int_a^b f(x) d x-T_n=-\frac{b-a}{12} h^2 f^{\prime \prime}(\eta), a<\eta<b
$$
其中 $h=\frac{b-a}{n}$.
证明每个小区间上
$$
R\left(f, T_n\right)=-\frac{h^3}{12} \sum_{k=0}^{n-1} f^{\prime \prime}\left(\eta_k\right), x_k<\eta_k<x_{k+1}
$$
由于 $f^{\prime \prime}(x)$ 连续, 则存在 $\eta \in[a, b]$,
$$
\frac{1}{n} \sum_{k=0}^{n-1} f^{\prime \prime}\left(\eta_k\right)=f^{\prime \prime}(\eta)
$$}

\theorem[5.4]{若 $f(x) \in C^4[a, b]$, 则存在 $\eta \in[a, b]$,
$$
R\left(f, S_n\right)=\int_a^b f(x) d x-T_n=-\frac{b-a}{2880} h_1^4 f^{(4)}(\eta)=-\frac{b-a}{180} h^4 f^{(4)}(\eta)
$$
其中 $h_1=2 h, h=\frac{b-a}{2 n}$.}






















% \begin{figure}[p]

%     \centerline{\includegraphics[width=1.2\linewidth,height=1.1\textheight]{name}}
%     \caption{课上习题}
%     \label{figure}

%\end{figure}



% \bibliographystyle{IEEEtran}
% \bibliography{reference}



\end{document}