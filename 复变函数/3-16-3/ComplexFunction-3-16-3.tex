\documentclass[12pt, a4paper, oneside]{ctexart}
\usepackage{amsmath, amsthm, amssymb, bm, graphicx, hyperref, geometry, mathrsfs,color}

\title{\huge\textbf{Cauchy'Theorem and Its Application}}
\author{luojunxun}
\date{\today}
\linespread{2}%行间距
\geometry{left=2cm,right=2cm,top=2cm,bottom=2cm}%设置页面
\CTEXsetup[format={\Large\bfseries}]{section}%section左对齐

%定义环境
\newenvironment{Def}[1][def-name]{\par\noindent{\textit{(#1):}\small}}{\\\par}
\newenvironment{theorem}[1][Theorem-name]{\par\noindent \textbf{Theorem #1:}\textit}{\\\par}
\newenvironment{corollary}[1][corollary-name]{\par\noindent \textbf{Corollary #1:}\textit}{\\\par\vspace*{15pt}}
\newenvironment{lemma}[1][lemma-name]{\par\noindent \textbf{Lemma #1:}\textbf}{\\\par}
\renewenvironment{proof}{\par\noindent{\textit{Proof:}\small}}{\\\par}
\newenvironment{example}[1][example-name]{\par{\textbf{Example:}}}{\\\par}
\newenvironment{say}{\center{\textit{summary:}}}{\\\par}
\newenvironment{note}[1][note-name]{\par\textit{#1:}}{\\\par}


\begin{document}
\maketitle
\section*{Goursat'theorem}
\theorem[1.1]{If $\Omega$ is an open set in $\mathbb{C}$, and $T \subset \Omega$ a triangle whose interior is also contained in $\Omega$, then
$$
\int_T f(z) d z=0
$$
whenever $f$ is holomorphic in $\Omega$.}

\corollary[1.2]{If $f$ is holomorphic in an open set $\Omega$ that contains a rectangle $R$ and its interior, then
$$
\int_R f(z) d z=0
$$}


\section*{Local existence of primitives and Cauchy's theorem ina disc}
\theorem[2.1]{A holomorphic function in an open disc has a primitive in that disc.}
\corollary[2.2]{A holomorphic function f in an open disc $\Omega$ ,$\gamma\subset\Omega$ is closed,then:
$\int_\gamma f(z)dz=0$}

\theorem[2.2(Cauchy's theorem for a disc)]{If $f$ is holomorphic in a disc, then
$$
\int_\gamma f(z) d z=0
$$
for any closed curve $\gamma$ in that disc.\\  
.$\qquad$Proof. Since $f$ has a primitive, we can apply Corollary 3.3 of Chapter 1 .}
\corollary[2.3]{Suppose $f$ is holomorphic in an open set containing the circle $C$ and its interior. Then
$$
\int_C f(z) d z=0
$$}

\section*{ Evaluation of some integrals}
1.$e^{-\pi \xi^2}=\int_{-\infty}^{\infty} e^{-\pi x^2} e^{-2 \pi i x \xi} d x;\xi=0 \Rightarrow 1=\int_{-\infty}^{\infty} e^{-\pi x^2} dx$\\
2.$\int_0^{\infty} \frac{1-\cos x}{x^2} d x=\frac{\pi}{2}$\\

\section*{Cauchy’s integral formulas}
{\center{
\theorem[4.1]{Suppose $f$ is holomorphic in an open set that contains the closure of a disc $D$. 
If $C$ denotes the boundary circle of this disc with the positive orientation, then:
\[f(z)=\frac{1}{2 \pi i} \int_C \frac{f(\zeta)}{\zeta-z} d \zeta \quad \text { for any point } z \in D\]}
}}
\note[4.1]{$while\;\zeta\in \Omega-\overline{D}:\int_C \frac{f(\zeta)}{\zeta-z} d \zeta=0$}

\corollary[4.2]{ If $f$ is holomorphic in an open set $\Omega$, then $f$ has infinitely many complex derivatives in $\Omega$. Moreover, if $C \subset \Omega$ is a circle whose interior is also contained in $\Omega$, then
$$
f^{(n)}(z)=\frac{n !}{2 \pi i} \int_C \frac{f(\zeta)}{(\zeta-z)^{n+1}} d \zeta
$$
for all $z$ in the interior of $C$.}

\corollary[4.3(Cauchy inequalities)]{
If $f$ is holomorphic in an open set that contains the closure of a disc $D$ centered at $z_0$ and of radius $R$, then
$$
\left|f^{(n)}\left(z_0\right)\right| \leq \frac{n !\|f\|_C}{R^n}
$$
where $\|f\|_C=\sup _{z \in C}|f(z)|$ denotes the supremum of $|f|$ on the boundary circle $C$.
}

\theorem[4.4]{Suppose $f$ is holomorphic in an open set $\Omega$. If $D$ is a disc centered at $z_0$ and whose closure is contained in $\Omega$, then $f$ has a power series expansion at $z_0$
\[
f(z)=\sum_{n=0}^{\infty} a_n\left(z-z_0\right)^n
\]
for all $z \in D$, and the coefficients are given by
\[
a_n=\frac{f^{(n)}\left(z_0\right)}{n !} \quad \text { for all } n \geq 0 .
\]}

\corollary[4.5 Liouville's theorem]{If f is entire and bounded, then f is constant.}










% \begin{figure}[p]

%     \centerline{\includegraphics[width=1.2\linewidth,height=1.1\textheight]{name}}
%     \caption{课上习题}
%     \label{figure}

%\end{figure}



% \bibliographystyle{IEEEtran}
% \bibliography{reference}



\end{document}