\documentclass[12pt, a4paper, oneside]{ctexart}
\usepackage{amsmath,extarrows, amsthm, amssymb, bm, graphicx, hyperref, geometry, mathrsfs,color}

\title{\huge\textbf{Meromorphic Functions and the Ldgarithm}}
\author{luojunxun}
\date{\today}
\linespread{2}%行间距
\geometry{left=2cm,right=2cm,top=2cm,bottom=2cm}%设置页面
\CTEXsetup[format={\Large\bfseries}]{section}%section左对齐

%定义环境
\newenvironment{Def}[1][def-name]{\par\noindent{\textit{(#1):}\small}}{\\\par}
\newenvironment{theorem}[1][Theorem-name]{\par\noindent \textbf{Theorem #1:}\textit}{\\\par}
\newenvironment{corollary}[1][corollary-name]{\par\noindent \textbf{Corollary #1:}\textit}{\\\par\vspace*{15pt}}
\newenvironment{lemma}[1][lemma-name]{\par\noindent \textbf{Lemma #1:}\textbf}{\\\par}
\renewenvironment{proof}{\par\noindent{\textit{Proof:}\small}}{\\\par}
\newenvironment{example}[1][example-name]{\par{\textbf{Example:}}}{\\\par}
\newenvironment{say}{\center{\textit{summary:}}}{\\\par}
\newenvironment{note}[1][note-name]{\par\textit{#1:}}{\\\par}


\begin{document}
\maketitle
\Def[Singularitiy]{$z_0$ is a ingularitiy of f while f has no defination at $z_0$}

\Def[Types of Singularities]{1.Removable Singularitiy:0 for $f(z)=\frac{\sin z}{z}$\qquad 2.Pole:0 for $f(z)=\frac{1}{z}$\par$
$3.Essential Singularity:0 for $e^{1/z}$}

\theorem[Laurent-serie]{$f(z) =f(z_0) +\sum\limits_{n=-\infty}^{-1}a_{n}(z-z_0)^n+\sum\limits_{n=0}^\infty (z-z_0)^n$ 
the first part is called the Principal-Part, the second part is called the Analytic-Part.espacially $a_{-1}$ is called the residue of f at $z_0$}\vspace*{5pt}

\theorem[1.1]{f hol on $\Omega$, $f(z_0)=0\Rightarrow\exists u=O(z_0,r)\;\exists g(z)$ defines on u,$\exists n>0\;s.t.f(z)=(z-z_0)^ng(z)$ the n called the order of $z_0$}

\theorem[1.2]{f hol on $\Omega$,$z_0$ is a pole of f $\Rightarrow\exists h(z)\;hol\;on\;\Omega,\exists n>0\;s.t.f(z)=(z-z_0)^{-n}h(z)$ n called the order of pole $z_0$}

\theorem[1.3]{if f has a pole of order n$\Rightarrow f(z) = \frac{a_{-n}}{(z-z_0)^n}+\cdots+\frac{a_{-1}}{z-z_0}+g(z)$ g is hol on a neiborhood of $z_0$}

\theorem[1.4]{$res_{z_0}f=\lim\limits_{z\to z_0}(z-z_0)f(z)=\lim\limits_{z\to z_0}\frac{1}{(n-1)!}\frac{d^n}{dz^n}(z-z_0)^nf(z)$}\vspace*{10pt}


\section*{The residue formula}

\theorem[2.1]{$z_0$ is a pole of f and f hol on $\Omega-\{z_0\},z_0\in Int(C) ,\;C\cup Int(C) \subset\Omega \quad \Rightarrow res_{z_0}f = \frac{1}{2\pi i}\int_C f(z)dz$}

\corollary[2.2]{$\int_Cf(z)dz=2\pi i\sum\limits_{k = 1}^Nres_{z_k}f$}

\corollary[2.3]{$\int_\gamma f(z)dz=2\pi i\sum\limits_{k = 1}^Nres_{z_k}f$}













% \begin{figure}[p]

%     \centerline{\includegraphics[width=1.2\linewidth,height=1.1\textheight]{name}}
%     \caption{课上习题}
%     \label{figure}

%\end{figure}



% \bibliographystyle{IEEEtran}
% \bibliography{reference}



\end{document}