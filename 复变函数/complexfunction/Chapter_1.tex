\documentclass[12pt, a4paper, oneside]{ctexart}
\usepackage{hyperref,amsmath,extarrows, amsthm, amssymb, bm, graphicx, hyperref, geometry, mathrsfs,color}

\title{\huge\textbf{Chapter1}}
\author{luojunxun}
\date{\today}
\linespread{2}%行间距
\geometry{left=2cm,right=2cm,top=2cm,bottom=2cm}%设置页面
\CTEXsetup[format={\Large\bfseries}]{section}%section左对齐

%定义环境
\newenvironment{Def}[1][def-name]{\par\noindent{\textit{(#1):}\small}}{\\\par}
\newenvironment{theorem}[1][Theorem-name]{\par\noindent \textbf{Theorem #1:}\textit}{\\\par}
\newenvironment{corollary}[1][corollary-name]{\par\noindent \textbf{Corollary #1:}\textit}{\\\par\vspace*{15pt}}
\newenvironment{proposition}[1][proposition-name]{\par\noindent \textbf{proposition #1:}\textit}{\\\par\vspace*{15pt}}
\newenvironment{lemma}[1][lemma-name]{\par\noindent \textbf{Lemma #1:}\textbf}{\\\par}
\renewenvironment{proof}{\par\noindent{\textit{Proof:}\small}}{\\\par}
\newenvironment{example}[1][example-name]{\par{\textbf{Example:}}}{\\\par}
\newenvironment{say}{\center{\textit{summary:}}}{\\\par}
\newenvironment{note}{\par\textit{ }}{\\\par}
\newcommand{\qie}{\enspace\&\enspace}


\begin{document}
\maketitle

\section*{Complex number and complex plane}

\theorem[1.1]{C, the complex numbers is complete}

\note{a set is complete is to say $X\overset{closed}{\subset} Y$ , $(Y,\rho)$ is Metric space .for any point $x\in X$,x is limit point}

\theorem[1.2]{The set $\Omega\subset C$ is compact iff every sequence $\{z_n\}\subset \Omega$ has a subsequence that converges to a point $z_0$ in $\Omega$}

\note{the most important is $z_0 \in \Omega$, for example for $\Omega = (0,1]\times [0,1],z_n = 1/n+i/2$, it's obviously $z_n\to i/2$ but $i/2\notin\Omega$}

\theorem[1.3]{A set $\Omega$ is compact iff every open covering has a finite subcovering that covers $\Omega$ }

\proposition[1.4]{if $\Omega_1\supset\Omega_2\supset\cdots\supset\Omega_n\supset\cdots$ is a sequence of non-empty compact set in C 
    with the property that \[diam(\Omega_n)\to 0\;as\;n\to\infty \]
    then there exists a unique point $\omega\in C$ s.t. $\omega\in\Omega_n$ for all n 
    }

\section*{Function of the complex plane}

\theorem[2.1]{A continuous function f on a compace set $\Omega$ is abounded and attains a maximun and minimun in $\Omega$}

\note{Holomophic Function:f hol at z iff $\lim\limits_{h\to 0}\frac{f(z+h)-f(z)}{h}$ converges to a limit\\
    f is said to be hol on $\Omega$ iff f hol at every point of it\\
    for example $f(z) = \frac{1}{z}$ isn't hol at (0,0);any polynomial hol in C,$f(z)=\overline{z}$ isn't hol since $\lim\limits_{h\to 0}\frac{f(z+h)-f(z)}{h}=\frac{\overline
    h}{h}$ has no limit}

\proposition[2.2]{$f,g$ hol on $\Omega$:(1):f+g hol, (f+g)'=f'+g' ; (2):fg hol, (fg)'=f'g+g'f ; (3):(f/g) hol where g novanish $(f/g)' = (f'g-g'f)/g^2$ and (4): $\Omega\overset{f}{\rightarrow}U\overset{g}{\rightarrow}C,g(f(z))$ hol on $\Omega$}

\proposition[2.3]{if f is hol at $z_0$ then :$\frac{\partial f}{\partial \overline{z}}f(z_0) = 0,f'(z_0)=\frac{\partial f}{\partial z}(z_0)=2\frac{\partial u}{\partial z}(z_0)$}

\proposition[2.4]{suppose f=u+iv is a complex-value function defined on $\Omega$ ,if u,v sre continuesly differentiable on $\Omega$ and satisify the Cauchy-Riemann Equations on $\Omega$, then f is hol on $\Omega$ and $f'(z) = \frac{\partial f}{\partial z}$}

\theorem[2.5]{Given a power seties $\sum\limits_{n=0}^{\infty} a_nz^n$ there exists $0\leq R\leq \infty$ s.t.\\
    (1):if |z|<R,the series converges absolutely\\
    (2):if |z|>R,the series diverges\\
    while |z|=R the situation need to discuss 
}

\note{$\frac{1}{R} = \limsup |a_n|^{\frac{1}{n}}$,we call R "radius of convergence",|z|<R "the disc of convergence"}

\theorem[2.6]{the power series $f(z) = \sum\limits_{n=0}^{\infty} a_nz^n$ defines a hol function in it's disc of convergence.the derivative of f is also a power serier for f that's
\[f'(z) = \sum\limits_{n=0}^\infty na_nz^{n-1}\]
moverover f' has the same radius of convergence as f}

\corollary[2.7]{A power series is infinitely complex differentiable in it's disc of convergence and the higher derivatives are also seties obtained by termwise differentiation}

\note{f is Holomophic iff f is Analytic iff f has a power series expansion}

\section*{integration along curves}

\proposition[3.1]{integration of continuous functions over cuvers satisfies the followling properties:
(i) It is linear, that is, if $\alpha, \beta \in \mathbb{C}$, then
$$
\int_\gamma(\alpha f(z)+\beta g(z)) d z=\alpha \int_\gamma f(z) d z+\beta \int_\gamma g(z) d z
$$
(ii) If $\gamma^{-}$is $\gamma$ with the reverse orientation, then
$$
\int_\gamma f(z) d z=-\int_{\gamma^{-}} f(z) d z
$$
(iii) One has the inequality
$$
\left|\int_\gamma f(z) d z\right| \leq \sup _{z \in \gamma}|f(z)| \cdot \text { length }(\gamma)
$$
}

\theorem[3.2]{if a function f has a primitive F in $\Omega$ and $\gamma$ is a curve in it that begins $w_1$ and ends $w_2$ then:
\[\int_\gamma f(z)dz = F(w_2)-F(w_1)\]
it's same like Newton-Leibniz Formula}

\corollary[3.3]{ If $\gamma$ is a closed curve in an open set $\Omega$, and $f$ is continuous and has a primitive in $\Omega$, then
$$
\int_\gamma f(z) d z=0 .
$$
This is immediate since the end-points of a closed curve coincide.}



















\end{document}