\documentclass[12pt, a4paper, oneside]{ctexart}
\usepackage{hyperref,amsmath, extarrows, amsthm, amssymb, bm, graphicx, hyperref, geometry, mathrsfs,color}

\title{\huge\textbf{Chapter3-Meromorphic Functions and the Logarithm}}
\author{luojunxun}
\date{\today}
\linespread{2}%行间距
\geometry{left=2cm,right=2cm,top=2cm,bottom=2cm}%设置页面
\CTEXsetup[format={\Large\bfseries}]{section}%section左对齐

%定义环境
\newenvironment{Def}[1][def-name]{\par\noindent{\textit{(#1):}\small}}{\\\par}
\newenvironment{theorem}[1][Theorem-name]{\par\noindent \textbf{Theorem #1:}\textit}{\\\par}
\newenvironment{corollary}[1][corollary-name]{\par\noindent \textbf{Corollary #1:}\textit}{\\\par\vspace*{15pt}}
\newenvironment{proposition}[1][proposition-name]{\par\noindent \textbf{proposition #1:}\textit}{\\\par\vspace*{15pt}}
\newenvironment{lemma}[1][lemma-name]{\par\noindent \textbf{Lemma #1:}\textbf}{\\\par}
\renewenvironment{proof}{\par\noindent{\textit{Proof:}\small}}{\\\par}
\newenvironment{example}[1][example-name]{\par{\textbf{Example:}}}{\\\par}
\newenvironment{say}{\center{\textit{summary:}}}{\\\par}
\newenvironment{note}{\par\textit{ }}{\\\par}
\newcommand{\qie}{\enspace\&\enspace}

\begin{document}
\maketitle

\section*{Zeros and poles}
\theorem[1.1]{Suppose that f is holomorphic in a connected open set Ω,has a zero at a point $z_0\in\Omega$, and does not vanish identically in Ω. Then
there exists a neighborhood $U\subset\Omega$ of $z_0$, a non-vanishing holomorphic function g on U, and a unique positive integer n such that;
\[
    f(z) = (z-z_0)^ng(z)
\]for all $z\in U$}

\note{we say f has zero of order n at $z_0$(n=1,say it' simple)}

\theorem[1.2]{If f has a pole at $z_0\in\Omega$, then in a neighborhood of that point there exist a non-vanishing holomorphic function h and a unique
positive integer n such that:
\[
    f(z) = (z-z_0)^{-n}h(z)
\]}

\note{The integer n is called the order of the pole(n=1,say it' simple)}

\theorem[1.3]{if f has a pole of order n at $z_0$,then\[
    f(z)=\frac{a_{-n}}{\left(z-z_0\right)^n}+\frac{a_{-n+1}}{\left(z-z_0\right)^{n-1}}+\cdots+\frac{a_{-1}}{\left(z-z_0\right)}+G(z)  
\]
where G is a holomorphic function in a neighborhood of $z_0$}

\note{the sum $P(z)=\frac{a_{-n}}{(z-z_0)^n}+\cdots+\frac{a_{-1}}{z-z_0}$ is called the Principal part of f at $z_0$.
the coefficient $a_{-1} $is the Residue of f at that pole ($res_{z_0}f=a_{-1}$)}

\note{In fact we can write $f(z)=P(z)+G(z)$,G is at most an infinite term series.Consider the integral along $C=C(z_0)$,we have\[
    \int_C f(z)dz = \int_C P(z)dz+\int_C G(z)dz
\]
for the second part,its value is obviously zero.for the first part,the value is $2\pi ia_{-1}$.}

\theorem[1.4]{if f has a pole of order n at $z_0$,then\[
  res_{z_0}f=\lim\limits_{z\to z_0}\frac{1}{(n-1)!}(\frac{d}{dz})^{n-1}(z-z_0)^nf(z)  
\]}

\section*{The residue formula}

\theorem[2.1]{Suppose that f is holomorphic in an open set containing a circle C and its interior, except for a pole at $z_0$ inside C. Then
\[
    \int_Cf(z)dz = 2\pi i res_{z_0}f.   
\]}

\corollary[2.2]{Suppose that f is holomorphic in an open set containing a circle C and its interior, except for poles at the points $z_1,...,z_N$ 
inside C. Then:
\[
    \int_Cf(z)dz = 2\pi i\sum_{k=1}^Nres_{z_k}f.    
\]}

\corollary[2.3(Residue formula)]{Suppose that f is holomorphic in an open set containing
a toy contour $\gamma$ and its interior, except for poles at the points $z_1,...,z_N$
inside $\gamma$. Then\[
    \int_\gamma f(z)dz = 2\pi i\sum_{k=1}^Nres_{z_k}f.
\]}

\section*{Singularities and Meromorphic functions}

\theorem[3.1 Riemann's theorem on removable singularities]{Suppose that f is holomorphic in an open set Ω except possibly at a point
$z_0$ in Ω. If f is bounded on $\Omega - \{z_0\}$, then $z_0$ is a removable singularity.}

\href{https://zhuanlan.zhihu.com/p/253065692}{Proof of 3.1 use the Taylor series of function $F(z)=(z-z_0)^2f(z)$}

\corollary[3.2]{Suppose that f has an isolated singularity at the point $z_0$. Then $z_0$ is a pole of f if and only if $|f(z)|\to\infty , z \to z_0.$}

\proof[3.2]{The necessity is obvious.Conversely,$1/f$ bounded on the neighborhood of $z_0$ and vanishs at $z_0$.Suppose the order of zero is n,that's
$1/f=\sum\limits_{k=n}a_k(z-z_0)^k$,so we have \[f(z) = \frac{1}{(z-z_0)^n\times \sum\limits_{k=n}^\infty a_k(z-z_0)^{k-n}}=\frac{1}{(z-z_0)^n}\frac{1}{a_n+a_{n+1}(z-z_0)+\cdots}=\frac{1}{(z-z_0)^n}g(z)\]\\
where $g(z)\neq 0$ on the neighborhood.that's what we need}

\note{Isolated singularities belong to one of three categories:\par
1.Removeable singularity(f is bounded near $z_0$)\par
2.Pole singularity($|f|\to\infty\;as\;z\to z_0$)\par
3.Essential singularity(Singularities those are not Pole or Removeable singularities)}

\theorem[3.3 Casorati-Weistrass]{Suppose f is holomorphic in the punctured disc $D_r(z_0)-\{z_0\}$ and has an essential singularity at $ z_0$.
Then, the image of $D_r(z_0)-\{z_0\}$  under f is dense in the complex plane.}

\note{f is dense is to say:$\forall w\in C,\forall\epsilon,\delta>0,\exists z,|z-z_0|<\delta \; s.t.|f(z)-w|<\epsilon$}

\note{In fact, Picard proved a much stronger result. He showed that under the hypothesis of the above theorem, the function f takes on every complex value infinitely many times with at most one exception}

\Def[Meromorphic function]{We now turn to functions with only isolated singularities that are poles. A function $f$ on an open set $\Omega$ is meromorphic if there exists a sequence of points $\left\{z_0, z_1, z_2, \ldots\right\}$ that has no limit points in $\Omega$, and such that
(i) the function $f$ is holomorphic in $\Omega-\left\{z_0, z_1, z_2, \ldots\right\}$, and
(ii) $f$ has poles at the points $\left\{z_0, z_1, z_2, \ldots\right\}$.\par
In conclusion, f is meromorphic is to say instead of poles f has no other singularities(removable singularity is allowed since we can replace it with a complex num)}

\theorem[3.4]{the meromorphic function f ($f:\overline{C}\to D$) in the extended complex plane ($\overline{C}=C\cup \{\infty\}$) are the rational functions}

\href{https://zhuanlan.zhihu.com/p/600909997}{theorem 3.4}

\section*{The argument principle and applications}

\theorem[4.1 Argument principle]{Suppose f is meromorphic in an open set containing a circle C and its interior. 
If f has no poles and never vanishes on C, then:
\[
    \frac{1}{2\pi i}\int_C \frac{f'(z)}{f(z)}dz=(number \;of \;zeros \;of \;f \;inside \;C) \;minus\; (number\; of\; poles \;of\; f\; inside C)\;
\]
where the zeros and poles are counted with their multiplicities.}

\corollary[4.2]{The above theorem holds for toy contours}

\theorem[4.3 Rouche's theorem]{Suppose that f and g are holomorphic in an open set containing a circle C and its interior. If
\[
    |f(z)|>|g(z)| \;forall\;z\in C,
\]
then f and f + g have the same number of zeros inside the circle C.}

\note{here |f|>|g| at C instead of at Int(C)!}

\theorem[4.4 Open mapping theorem]{If f is holomorphic and non-constant in a region Ω, then f is open.}

\note{A mapping is said to be open if it maps open sets to open sets.}

\theorem[4.5 Maximum modulus principle]{If f is a non-constant holomorphic function in a region Ω, then f cannot attain a maximum in Ω.}

\corollary[4.6]{Suppose that Ω is a region with compact (so bounded) closure $\overline{\Omega}$. If f is holomorphic on Ω and continuous on $\overline{\Omega}$. then
\[
    \sup _{z \in \Omega}|f(z)| \leq \sup _{z \in \bar{\Omega}-\Omega}|f(z)| .
\]
In fact, since $f(z)$ is continuous on the compact set $\bar{\Omega}$, then $|f(z)|$ attains its maximum in $\bar{\Omega}$; 
but this cannot be in $\Omega$ if $f$ is non-constant. If $f$ is constant, the conclusion is trivial.}

\section*{ Homotopies and simply connected domains}
\note{Loosely speaking, two curves are homotopic if one curve can be deformed into the other by a continuous transformation without ever leaving Ω}

\theorem[5.1]{ If $f$ is holomorphic in $\Omega$, then
\[
\int_{\gamma_0} f(z) d z=\int_{\gamma_1} f(z) d z
\]
whenever the two curves $\gamma_0$ and $\gamma_1$ are homotopic in $\Omega$.}

\note{A region Ω in the complex plane is simply connected if any two pair of curves in Ω with the same end-points are homotopic.}

\theorem[5.2]{Any holomorphic function in a simply connected domain has a primitive.}

\corollary[5.3]{If $f$ is holomorphic in the simply connected region $\Omega$, then
\[
\int_\gamma f(z) d z=0
\]
for any closed curve $\gamma$ in $\Omega$.}

\section*{The complex logarithm}

\href{https://zhuanlan.zhihu.com/p/422338793}{logarithm}

\theorem[6.1]{Suppose that $\Omega$ is simply connected with $1 \in \Omega$, and $0 \notin$ $\Omega$. Then in $\Omega$ there is a branch of the logarithm $F(z)=\log _{\Omega}(z)$ so that
\\(i) $F$ is holomorphic in $\Omega$,
\\(ii) $e^{F(z)}=z$ for all $z \in \Omega$,
\\(iii) $F(r)=\log r$ whenever $r$ is a real number and near 1 .
\\In other words, each branch $\log _{\Omega}(z)$ is an extension of the standard logarithm defined for positive numbers.}

\theorem[6.2]{If f is a nowhere vanishing holomorphic function in a simply connected region Ω, then there exists a holomorphic function g on Ω such that
\[
    f(z) = e^{g(z)}
\]
The function g(z) in the theorem can be denoted by log f(z), and determines a “branch” of that logarithm.}

\section*{Fourier series and harmonic functions}

\theorem[7.1]{The coefficients of the power series expansion of $f$ are given by
\[
a_n=\frac{1}{2 \pi r^n} \int_0^{2 \pi} f\left(z_0+r e^{i \theta}\right) e^{-i n \theta} d \theta
\]
for all $n \geq 0$ and $0<r<R$. Moreover,
\[
0=\frac{1}{2 \pi r^n} \int_0^{2 \pi} f\left(z_0+r e^{i \theta}\right) e^{-i n \theta} d \theta
\]
whenever $n<0$.}

\corollary[7.2]{(Mean-value property) If $f$ is holomorphic in a disc $D_R\left(z_0\right)$, then
\[
f\left(z_0\right)=\frac{1}{2 \pi} \int_0^{2 \pi} f\left(z_0+r e^{i \theta}\right) d \theta, \quad \text { for any } 0<r<R .
\]}

\corollary[7.3]{If $f$ is holomorphic in a disc $D_R\left(z_0\right)$, and $u=\operatorname{Re}(f)$, then
\[
u\left(z_0\right)=\frac{1}{2 \pi} \int_0^{2 \pi} u\left(z_0+r e^{i \theta}\right) d \theta, \quad \text { for any } 0<r<R .
\]}

\note{ every harmonic function in a disc is the real part of a holomorphic
function in that disc.( Exercise 12 in Chapter 2)}





\end{document}