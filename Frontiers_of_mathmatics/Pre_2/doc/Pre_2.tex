%# -*- coding:utf-8 -*-
\documentclass[10pt,aspectratio=169,mathserif]{beamer}		
%设置为 Beamer 文档类型,设置字体为 10pt,长宽比为16:9,数学字体为 serif 风格

%%%%-----导入宏包-----%%%%
\usepackage{zju}			%导入 zju 模板宏包
\usepackage{ctex}			%导入 ctex 宏包,添加中文支持
\usepackage{amsmath,amsfonts,amssymb,bm}   %导入数学公式所需宏包
\usepackage{color}			 %字体颜色支持
\usepackage{graphicx,hyperref,url}
\usepackage{metalogo}	% 非必须
\usepackage{ragged2e}
%% 上文引用的包可按实际情况自行增删
%%%%%%%%%%%%%%%%%%
\usepackage{fontspec}
\usepackage{xeCJK}
% \setCJKmainfont{Source Han Sans SC}



\beamertemplateballitem		%设置 Beamer 主题

%%%%------------------------%%%%%
\catcode`\。=\active         %或者=13
\newcommand{。}{.}				
%将正文中的“。”号转换为“.”。中文标点国家规范建议科技文献中的句号用圆点替代
%%%%%%%%%%%%%%%%%%%%%

%%%%----首页信息设置----%%%%
\title[exceptation maximization algorithm]{EM 算法及其推广}
\subtitle{--算法介绍和收敛性证明}			
%%%%----标题设置


\author[罗俊勋]{
  罗俊勋
}
  
\institute[IOPP]{
  School of Mathmatical\\ 
  Zhejiang University}
%%%%----机构信息

\date[Sept. 25 2024]{
  2024年11月06日}
%%%%----日期信息
  
\begin{document}

\begin{frame}
	\titlepage
\end{frame}				%生成标题页


\begin{frame}
%%  	\frametitle{提纲}
	\tableofcontents
\end{frame}				%生成提纲页

\section{EM 算法介绍}
\begin{frame}
  \frametitle{EM 算法介绍}
  \begin{block}{适用场景}
  适用于模型已知,参数未定的情况:如果没有隐变量就直接使用最大似然估计,若不然考虑使用 EM 算法。
  
  \begin{itemize}
    \item 统计男女生身高,已知其服从正态分布,但不同性别的均值和方差未知。(数据混在一起)
      \begin{itemize}
        \item 根据名字判断性别,但有些名字是中性的。
        \item 根据身高判断性别,但有些人的身高不符合性别特征。
      \end{itemize}
    
    \item 三枚质量不均的硬币 $A,B,C$ ,正面出现的结果为 $\pi,p,q$。每一次实验抛掷两次,第一次抛 $A$, 如果正面则抛 $B$,否则抛 $C$,只记录最后一次的结果。正面为 1,反面为 0。

  \end{itemize}
  \end{block}
\end{frame}

\begin{frame}
  \frametitle{EM 算法流程和 Q 函数定义}
  \begin{block}{算法流程}
    算法接受变量数据 $Y$, 隐变量数据 $Z$, 联合分布 $P(Y,Z|\theta)$, 条件分布 $P(Z|Y,\theta)$。输出参数 $\theta$。

    \begin{itemize}
      \item 初始化参数 $\theta^{(0)}$。
      \item $E$ 步: $Q(\theta,\theta^{(i)}) = \mathbb{E}_Z [\log P(Y,Z|\theta)|Y,\theta^{(i)}]$
      \item $M$ 步: $\theta^{(i+1)} = \arg\max\limits_{\theta} Q(\theta,\theta^{(i)})$
      \item 重复 $E,M$ 步骤,直到满足收敛条件
    \end{itemize}
  \end{block}
  
  \begin{block}{$Q$ 函数}
    完全数据的对数似然函数 $\log P(Y,Z|\theta)$ 关于给定观测数据 $Y$, 和当前参数 $\theta^{(i)}$, 下对未观测数据 $Z$ 的期望

    $$Q(\theta,\theta^{(i)}) = \mathbb{E}_Z [\log P(Y,Z|\theta)|Y,\theta^{(i)}]$$
  \end{block}
\end{frame}

\section{EM 算法导出}

\begin{frame}
  \frametitle {算法导出}
  要通过迭代求出 $L(\theta) = P(Y|\theta)$ 的极值,我们希望 $L(\theta^{(i+1)})\geq L(\theta^{(i)})$  
  
  \begin{align*}
    L(\theta)-L\left(\theta^{(i)}\right) & =\log \left(\sum_Z P\left(Z \mid Y, \theta^{(i)}\right) \frac{P(Y \mid Z, \theta) P(Z \mid \theta)}{P\left(Z \mid Y, \theta^{(i)}\right)}\right)-\log P\left(Y \mid \theta^{(i)}\right) \\
    & \geq \sum_Z P\left(Z \mid Y, \theta^{(i)}\right) \log \frac{P(Y \mid Z, \theta) P(Z \mid \theta)}{P\left(Z \mid Y, \theta^{(i)}\right)}-\log P\left(Y \mid \theta^{(i)}\right) \\
    & =\sum_Z P\left(Z \mid Y, \theta^{(i)}\right) \log \frac{P(Y \mid Z, \theta) P(Z \mid \theta)}{P\left(Z \mid Y, \theta^{(i)}\right) P\left(Y \mid \theta^{(i)}\right)}
  \end{align*}
  
  令 $B(\theta,\theta^{(i)}) = L(\theta^{(i)}) + \sum_Z P\left(Z \mid Y, \theta^{(i)}\right) \log \frac{P(Y \mid Z, \theta) P(Z \mid \theta)}{P\left(Z \mid Y, \theta^{(i)}\right) P\left(Y \mid \theta^{(i)}\right)}$
  则 $L(\theta^{(i)}) \geq B(\theta,\theta^{(i)})$, 当且仅当 $\theta = \theta^{(i)}$ 时取等
  
\end{frame}

\begin{frame}
  \frametitle {算法导出}
  当我们尝试去优化下界,也就是命 $\theta^{(i+1)} = \arg\max\limits_{\theta} B(\theta,\theta^{(i)})$, 时,有:

  \begin{align*}
    \theta^{(i+1)} & =\arg \max _\theta\left(L\left(\theta^{(i)}\right)+\sum_Z P\left(Z \mid Y, \theta^{(i)}\right) \log \frac{P(Y \mid Z, \theta) P(Z \mid \theta)}{P\left(Z \mid Y, \theta^{(i)}\right) P\left(Y \mid \theta^{(i)}\right)}\right) \\
    & =\arg \max _\theta\left(\sum_Z P\left(Z \mid Y, \theta^{(i)}\right) \log (P(Y \mid Z, \theta) P(Z \mid \theta))\right) \\
    & =\arg \max _\theta\left(\sum_Z P\left(Z \mid Y, \theta^{(i)}\right) \log P(Y, Z \mid \theta)\right) \\
    & =\arg \max _\theta Q\left(\theta, \theta^{(i)}\right)
  \end{align*}
  
  这就是我们在 $M$ 步中做的事


\end{frame}

\section{EM 算法收敛性证明}
\begin{frame}
  \frametitle {收敛性证明}
  % \begin{block}{$\{P(Y||theta^{(i)})\}_{i=1}^\infty$ 单增}
    因为
    $$
    P(Y|\theta) = \frac{ P(Y,Z|\theta) }{ P(Z|Y,\theta) }
    $$
    令
    $$
    H(\theta,\theta^{(i)}) = \sum\limits_{Z}P(Z|Y,\theta^{(i)})\log P(Z|Y,\theta) 
    $$
    并且
    $$
    Q(\theta,\theta^{(i)}) = \sum\limits_{Z}P(Z|Y,\theta^{(i)}) \log P(Z,Y|\theta) 
    $$
    那么对数似然函数
    \begin{align*}
      L(\theta) &= \log P(Y|\theta) = \log P(Y,Z|\theta) - \log P(Z|Y,\theta) \\
      &= \sum\limits_{Z} P(Z|Y,\theta^{(i)})\log P(Z,Y|\theta)  - P(Z|Y,\theta^{(i)})\sum\limits_{Z} \log P(Z|Y,\theta)  \\
      &= Q(\theta,\theta^{(i)}) - H(\theta,\theta^{(i)})
      \\
    \end{align*} 
  % \end{block}
  
\end{frame}

\begin{frame}
  \frametitle {收敛性证明}
  $L(\theta^{(i+1)})-L(\theta^{(i)}) = \left[ Q(\theta^{(i+1)}) - Q(\theta^{(i)}) \right] - \left[ H(\theta^{(i+1)}) - H(\theta^{(i)}) \right]$
  
  前半部分已经为非负,现在考虑后半部分

  $$
  \begin{aligned} 
    H\left(\theta^{(i+1)}, \theta^{(i)}\right)-H\left(\theta^{(i)}, \theta^{(i)}\right) & =\sum_Z\left(\log \frac{P\left(Z \mid Y, \theta^{(i+1)}\right)}{P\left(Z \mid Y, \theta^{(i)}\right)}\right) P\left(Z \mid Y, \theta^{(i)}\right) \\ 
      & \leqslant \log \left(\sum_Z \frac{P\left(Z \mid Y, \theta^{(i+1)}\right)}{P\left(Z \mid Y, \theta^{(i)}\right)} P\left(Z \mid Y, \theta^{(i)}\right)\right) \\ 
      & =\log \left(\sum_Z P\left(Z \mid Y, \theta^{(i+1)}\right)\right)=0
  \end{aligned}
  $$
  
  这就得到了$L(\theta^{(i+1)})\geq L(\theta^{(i)})$
  
  显然有 $L(\theta) \leq 1$ 根据单调有界定理,$\{L(\theta^{(i)})\}_{i=1}^\infty$ 收敛。
\end{frame}

\section{EM 算法举例}
\begin{frame}
  \frametitle{混合高斯}
  现有男女生共 100人,已知男女生升高分别服从正态分布。求分布的各个参数
  \begin{itemize}
    \item 初始化 $\theta_0 = (\mu_b,\mu_g,\sigma_b,\sigma_b)$
    \item E-step: 计算 $P(Z|Y,\theta)$
    \begin{itemize}
      \item 每个人是男生的概率: $\vec{P_b} = f_{\mu_b,\sigma_b}(Y)$
      \item 每个人是女生的概率: $\vec{P_g} = f_{\mu_g,\sigma_g}(Y)$
      \item 按概率大小更新 $Z$: $Z = \vec{P_b} > \vec{P_g}$
    \end{itemize}
    \item M-step: 这里求最大很简单,直接让估计值等于样本的均值和方差
    \begin{itemize}
      \item $\mu_b = \frac{\sum\limits_{Z == 1}Y}{\sum\limits_{Z == 1}1}$
      \item $\sigma_b = \sqrt{\frac{\sum\limits_{Z == 1} (Y - \mu_b)^2}{\sum\limits_{Z == 1} 1}}$
      \item $\mu_g = \frac{\sum\limits_{Z == 0} Y}{\sum\limits_{Z == 0} 1}$
      \item $\sigma_g = \sqrt{\frac{\sum\limits_{Z == 0} (Y - \mu_g)^2}{\sum\limits_{Z == 0} 1}}$
    \end{itemize}
    \item 重复 E,M 步骤,直到满足收敛条件
  \end{itemize}
\end{frame}

\section*{\centering \LARGE 结语}

\end{document}
