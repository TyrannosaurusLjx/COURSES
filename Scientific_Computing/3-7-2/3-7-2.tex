\documentclass[12pt, a4paper, oneside]{ctexart}
\usepackage{amsmath, amsthm, amssymb, bm, graphicx, hyperref, geometry, mathrsfs,color}

\title{\huge\textbf{插值理论}}
\author{luojunxun}
\date{\today}
\linespread{2}%行间距
\geometry{left=2cm,right=2cm,top=2cm,bottom=2cm}%设置页面
\CTEXsetup[format={\Large\bfseries}]{section}%section左对齐

%定义环境
\newenvironment{Def}[1][def-name]{\par\noindent{\textit{(#1):}\small}}{\\\par}
\newenvironment{theorem}[1][Theorem-name]{\par\noindent \textbf{Theorem #1:}\textit}{\\\par}
\newenvironment{lemma}[1][lemma-name]{\par\noindent \textbf{Lemma #1:}\textbf}{\\\par}
\renewenvironment{proof}{\par\noindent{\textit{Proof:}\small}}{\\\par}
\newenvironment{example}[1][example-name]{\par{\textbf{Example:}}}{\\\par}
\newenvironment{say}{\center{\textit{summary:}}}{\\\par}
\newenvironment{note}[1][note-name]{\par\textit{#1:}}{\\\par}



\begin{document}
\maketitle



\say{本章讨论的是如果有n个点$(x_i,f(x_i));(i=0,1,\cdots,n)$,如何找到一个函数$\phi(x)$去拟合函数$f(x)$,精确的穿过这n个点\\
$f(x)\text{称为被插值函数},\phi(x)\text{称为插值函数},x_i\text{称为插值节点}$\\$\phi(x_i)=f(x_i)\text{称为插值条件},[a,b]\text{称为插值区间},R(x)=f(x)-\phi(x)\text{称为插值余项}$
}\cite{chazhi}

\section*{多项式插值}
{
\Def[Lagrange插值多项式]{$\phi(x)=y_0l_0(x)+y_1l_1(x)+\cdots+y_nl_n(x)$\\\center{
    若n次多项式$l_j(x);(j=0,1,2\cdots,n)$在n+1个节点$x_0<x_1<\ldots<x_n$上满足:\\
    $l_k\left(x_j\right)=\left\{\begin{array}{ll}1, & j=k \\ 0, & j\neq k\end{array},(j, k=0,1, \ldots, n)\right.$
    则称这n+1个n次多项式为节点$x_0<x_1<\ldots<x_n$上的n次插值基函数.\\
    lagrange:$l_k(x)=\prod\limits_{i=0,i\neq k}^n\frac{x-x_i}{x_k-x_i}=\frac{\left(x-x_0\right) \ldots\left(x-x_{k-1}\right)\left(x-x_{k+1}\right) \ldots\left(x-x_n\right)}{\left(x_k-x_0\right) \ldots\left(x_k-x_{k-1}\right)\left(x_k-x_{k+1}\right) \ldots\left(x_k-x_n\right)}$\\
    由此可得:$L_n(x)=\sum\limits_{k=0}^n y_k l_k(x)$称为lagrange插值函数}}
}\\
\vspace*{15pt}

{
\note[Lagrange插值多项式性质和误差估计]{
    1.$L_n(x)$还可以写为$L_n(x)=\sum_{k=0}^n \frac{\omega_{n+1}(x)}{\left(x-x_k\right) \omega_{n+1}^{\prime}\left(x_k\right)}$\\其中$\omega_{n+1}=\left(x-x_0\right)\left(x-x_1\right) \ldots\left(x-x_n\right)$\\
    $\omega_{n+1}^{\prime}\left(x_k\right)=\left(x_k-x_0\right) \ldots\left(x_k-x_{k-1}\right)\left(x_k-x_{k+1}\right) \ldots\left(x_k-x_n\right)$\\
    2.$R(x)=f(x)-\phi(x)=\frac{f^{(n+1)}(\xi)}{(n+1)!}\omega_n(x);\xi\in[a,b]$\\
    3.次数小于n的多项式用n次多项式去做插值得到的插值函数就是原函数,从而没有误差;特别的,令$y\equiv 1,\text{从而有}:1\equiv\sum\limits_{k=1}^nl_k(x)$
    }\\
}\vspace*{15pt}

\paragraph*{Lagrange插值的弊端就是,根据上述n+1个点能都计算出所有的插值基函数,但是如果突然新增一个点,所有的插值基函数都要重新进计算一次.
因此我们引入Newton插值,他的优点就是每次新增一个新点时只需要在末尾增加一项即可,具有"继承"的特性}\vspace*{15pt}

\paragraph*{\lemma[均差]{
    $f[x_i]=f(x_i)\text{称为$f(x)$在}x_i\text{上的零阶均差}$\\
    $f\left[x_i, x_{i+1}\right]=\frac{f\left(x_{i+1}\right)-f\left(x_i\right)}{x_{i+1}-x_i}\text{称为f(x)在}x_i,x_{i+1}\text{上的一阶均差}$\\
    $f\left[x_i, x_j\right]=\frac{f\left(x_j\right)-f\left(x_i\right)}{x_j-x_i}\text{称为f(x)在}x_i,x_{j}\text{上的一阶均差}$\\
    $f\left[x_i, x_j, x_k\right]=\frac{f\left[x_j, x_k\right]-f\left[x_i, x_j\right]}{x_k-x_i}\text{称为f(x)在}x_i,x_j,x_k\text{上的二阶均差}$\\
    $f\left[x_0, x_1, \cdots x_k\right]=\frac{f\left[x_1, \cdots, x_k\right]-f\left[x_0, \cdots, x_{k-1}\right]}{x_k-x_0}\text{k阶均差}$\\
    $f\left[x_0, x_1, \cdots, x_k\right]=\sum_{i=0}^k \frac{f\left(x_i\right)}{\left(x_i-x_0\right)\left(x_i-x_1\right) \cdots\left(x_i-x_{i-1}\right)\left(x_i-x_{i+1}\right) \cdots\left(x_i-x_k\right)}\text{为f(x)的k阶均差(差商)}$
}}\cite{juncha}
\vspace*{15pt}


\Def[Newton插值多项式]{\\$N(x)=c_0+c_1(x-x_0)+c_2(x-x_0)(x-x_1)+\cdots+c_n(x-x_0)\cdots(x-x_{n-1})$\center{
    $\mathrm{N}_{\mathrm{n}}(\mathrm{x})=\mathrm{f}\left(\mathrm{x}_0\right)+\mathrm{f}\left[\mathrm{x}_0, \mathrm{x}_1\right]\left(\mathrm{x}-\mathrm{x}_0\right)+\mathrm{f}\left[\mathrm{x}_0 , \mathrm{x}_1, \mathrm{x}_2\right]
    \left(\mathrm{x}-\mathrm{x}_0\right)\left(\mathrm{x}-\mathrm{x}_1\right)+\cdots+\mathrm{f}\left[\mathrm{x}_0, \mathrm{x}_1, \cdots, \mathrm{x}_{\mathrm{n}}\right]\left(\mathrm{x}-\mathrm{x}_0\right)\left(\mathrm{x}-\mathrm{x}_1\right) \cdots\left(\mathrm{x}-\mathrm{x}_{\mathrm{n}-1}\right)$
}}\vspace*{15pt}

\note[Newton插值多项式性质和误差估计]{\\
    1.$R_n(x) =f(x)-P_n(x)=f\left[x_0, x_1, \cdots, x_n, x\right]\left(x-x_0\right)\left(x-x_1\right) \cdots\left(x-x_{n-1}\right)\left(x-x_n\right)$\\
    2.差商具有线性性若$f(x)=a \varphi(x)+b \psi(x)$,则$f\left[x_0, x_1, \cdots, x_k\right]=a \varphi\left[x_0, x_1, \cdots, x_k\right]+b \psi\left[x_0, x_1, \cdots, x_k\right]$\\
    3.由于插值多项式唯一,比较与拉格朗日的最高次项系数就有:$f\left[x_0, x_1, \cdots, x_n\right]=\sum_{i=0}^n \frac{f\left(x_i\right)}{\omega_{n+1}^{\prime}\left(x_i\right)}$\\
    4.差商对称性$f\left[x_0, x_1, \cdots, x_k\right]=f\left[x_{i_0}, x_{i_1}, \cdots, x_{i_k}\right]$
}\vspace*{15pt}





























% \begin{figure}[p]

%     \centerline{\includegraphics[width=1.2\linewidth,height=1.1\textheight]{name}}
%     \caption{课上习题}
%     \label{figure}

% \end{figure}






















\bibliographystyle{IEEEtran}
\bibliography{reference}



\end{document}