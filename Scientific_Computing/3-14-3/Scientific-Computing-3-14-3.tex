\documentclass[12pt, a4paper, oneside]{ctexart}
\usepackage{amsmath, amsthm, amssymb, bm, graphicx, hyperref, geometry, mathrsfs,color}

\title{\huge\textbf{多项式插值2}}
\author{luojunxun}
\date{\today}
\linespread{2}%行间距
\geometry{left=2cm,right=2cm,top=2cm,bottom=2cm}%设置页面
\CTEXsetup[format={\Large\bfseries}]{section}%section左对齐

%定义环境
\newenvironment{Def}[1][def-name]{\par\noindent{\textit{(#1):}\small}}{\\\par}
\newenvironment{theorem}[1][Theorem-name]{\par\noindent \textbf{Theorem #1:}\textit}{\\\par}
\newenvironment{lemma}[1][lemma-name]{\par\noindent \textbf{Lemma #1:}\textbf}{\\\par}
\renewenvironment{proof}{\par\noindent{\textit{Proof:}\small}}{\\\par}
\newenvironment{example}[1][example-name]{\par{\textbf{Example:}}}{\\\par}
\newenvironment{say}{\center{\textit{summary:}}}{\\\par}
\newenvironment{note}[1][note-name]{\par\textit{#1:}}{\\\par}


\begin{document}
\maketitle


\theorem[定理]{:
对任意的 $x$, 若 $x \neq x_i, i=0,1,2, \cdots, n$, 则
$$
\begin{aligned}
f(x)= & f\left(x_0\right)+f\left[x_0, x_1\right]\left(x-x_0\right) \\
& +f\left[x_0, x_1, x_2\right]\left(x-x_0\right)\left(x-x_1\right)+\cdots \\
& +f\left[x_0, x_1, \cdots, x_n\right]\left(x-x_0\right)\left(x-x_1\right) \cdots\left(x-x_{n-1}\right) \\
& +f\left[x_0, x_1, \cdots, x_n, x\right]\left(x-x_0\right)\left(x-x_1\right) \cdots\left(x-x_{n-1}\right)\left(x-x_n\right)
\end{aligned}
$$
    }

\section*{Hermite插值}

问题: 给定函数 $f(x)$ 在节点 $x_0, x_1, \cdots, x_n$ 处的函数值及一阶导 数值, 求 $2 n+1$ 次多项式 $H_{2 n+1}(x)$ 满足条件
$$
H_{2 n+1}\left(x_i\right)=f\left(x_i\right), \quad H_{2 n+1}^{\prime}\left(x_i\right)=f^{\prime}\left(x_i\right), \quad i=0,1, \cdots, n .
$$

\begin{center}
    求解:$H_{2 n+1}(x)=\sum_{i=0}^n y_i A_i(x)+\sum_{i=0}^n y_i^{\prime} B_i(x)$(其中A,B是待定的次数不超过2n+1次的多项式)\\并且有
    $\begin{aligned} & A_i\left(x_j\right)=\delta_{i j}, \quad A_i^{\prime}\left(x_j\right)=0, 
    \quad i, j=0,1, \cdots, n \\ & B_i\left(x_j\right)=0, \quad B_i^{\prime}\left(x_j\right)=\delta_{i j}, \quad i, j=0,1, \cdots, n \text {. } \\ & \end{aligned}$\\
    解得:$\begin{aligned} B_i(x) & =\frac{\left(x-x_0\right)^2\left(x-x_1\right)^2 \cdots\left(x-x_{i-1}\right)^2\left(x-x_i\right)\left(x-x_{i+1}\right)^2 
        \cdots\left(x-x_n\right)^2}{\left(x_i-x_0\right)^2\left(x_i-x_1\right)^2 \cdots\left(x_i-x_{i-1}\right)^2\left(x_i-x_{i+1}\right)^2 \cdots\left(x_i-x_n\right)^2} \\
         & =\frac{\omega_n^2(x)}{\left(x-x_i\right)\left[\omega_n^{\prime}\left(x_i\right)\right]^2}=\left(x-x_i\right) l_i^2(x) .\end{aligned}$\\
         $\begin{aligned} A_i(x) & =\left(1-2\left(x-x_i\right) l_i^{\prime}\left(x_i\right)\right) \frac{\omega_n^2(x)}{\left(x-x_i\right)^2\left[\omega_n^{\prime}\left(x_i\right)\right]^2}
             \\ & =\left(1-2\left(x-x_i\right) l_i^{\prime}\left(x_i\right)\right) l_i^2(x) .\end{aligned}$\\
        最后得到:$\begin{aligned} H_{2 n+1}(x) & =\sum_{i=0}^n\left(y_i\left(1-2\left(x-x_i\right) \sum_{\substack{j=0 \\ j \neq i}}^n \frac{1}{x_i-x_j}\right)+y_i^{\prime}\left(x-x_i\right)\right) 
            \prod_{\substack{j=0 \\ j \neq i}}^n\left(\frac{x-x_j}{x_i-x_j}\right)^2 \\ & =\sum_{i=0}^n\left(y_i+\left(x_i-x\right)\left(2 y_i \sum_{\substack{j=0 \\ j \neq i}}^n \frac{1}{x_i-x_j}-y_i^{\prime}\right)\right)
            \left(\prod_{\substack{j=0 \\ j \neq i}}^n \frac{x-x_j}{x_i-x_j}\right)^2\end{aligned}$
\end{center}\vspace*{15pt}

误差估计:定理: 设 $x_0, x_1, \cdots, x_n$ 是区间 $[a, b]$ 上的 $n+1$ 个互不相同的 点, $f(x) \in C^{2 n+2}[a, b]$, 且 $f\left(x_i\right)=y_i$,
$f^{\prime}\left(x_i\right)=y_i^{\prime}(i=0,1, \cdots, n), H_{2 n+1}(x)$ 是Hermite揷值多项式. 则 对每个 $x \in[a, b]$, 存在 $\xi \in(a, b)$, 使得
$$
\begin{gathered}
f(x)-H_{2 n+1}(x)=\frac{f^{(2 n+2)}(\xi)}{(2 n+2) !} \omega_n^2(x) . \\
\omega_n(x)=\left(x-x_0\right)\left(x-x_1\right) \cdots\left(x-x_n\right) .
\end{gathered}
$$















% \begin{figure}[p]

%     \centerline{\includegraphics[width=1.2\linewidth,height=1.1\textheight]{name}}
%     \caption{课上习题}
%     \label{figure}




%\end{figure}




% \bibliographystyle{IEEEtran}
% \bibliography{reference}



\end{document}