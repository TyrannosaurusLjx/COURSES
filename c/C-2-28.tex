\documentclass[12pt, a4paper, oneside]{ctexart}
\usepackage{amsmath, amsthm, amssymb, bm, graphicx, hyperref, mathrsfs,color}

\title{\huge\textbf{引言}}
\author{luojunxun}
\date{\today}
\linespread{2}%行间距
\CTEXsetup[format={\Large\bfseries}]{section}%section左对齐

%定义环境
\newenvironment{Def}[1][def-name]{\par\noindent{\textit{(#1):}\small}}{\\\par}
\newenvironment{theorem}[1][Theorem-name]{\par\noindent \textbf{Theorem #1:}\textit}{\\\par}
\newenvironment{lemma}[1][lemma-name]{\par\noindent \textbf{Lemma #1:}\textbf}{\\\par}
\renewenvironment{proof}{\par\noindent{\textit{Proof:}\small}}{\\\par}
\newenvironment{example}[1][example-name]{\par{\textbf{Example:}}}{\\\par}
\newenvironment{say}{\center{\textit{summary:}}}{\\\par}
\newenvironment{note}[1][note-name]{\par\textit{#1:}}{\\\par}


\begin{document}
\maketitle

C语言的标识符由字母数字下划线组成,其中第一个字符必须是字母或者下划线\\

C语言生成的代码质量高,运行效率高\\

C语言源程序文件通过了编译、连接之后,生成一个后缀为.EXE的文件\\

程序与数据一样,共同存储在存储器中。当程序要运行时,当前准备运行的指令从内存被调入CPU中,由CPU处理这条指令.
这种将程序与数据共同存储的思想就是目前绝大多数计算机采用的(冯诺依曼)模型的存储程序概念.\\

下列(可移植性差)不是C语言所表现出来的不足之处:\\
数据类型坚持不严格\\表达式容易出现二义性\\可以执行查\\不能自动检查数据越界\\

C语言从main函数开始执行而且main函数可以放在任意位置\\

单目运算符是指只接受一个操作数的操作符,包括赋值运算符(=)、
% 算术运算符(+ 、-、*、/)、逻辑运算符(|| 、&& 、!,位逻辑运算符(& 、| 、^ 、~)、位移运算符(>>、<<)、关系运算符(> 、< 、==)、自增自减运算符(++ 、–)\\

算术运算注意点:1.两个整型数据做除法的时候,结果一定是整数;2.求余运算不能用于实数之间,只能在整数之间用;3.双目运算符的操作数的类型要相同,若否,则系统会自动转换\\

C程序中,用一对大括号 括起来的多条语句称为复合语句,复合语句在语法上被认为是一条语句。\\

在C语言中,仅由一个分号( )构成的语句称为空语句,它什么也不做。\\


这里注意整数除以后还是整数,所以结果是0\\

算术运算>关系运算>赋值运算\\

C语言中,字符型数据的值就是其在ASCII字符集中的次序值,即ASCII码\\

使用stdlib.h中的realloc来动态分配数组\\

如果函数的返回类型是指针,则可以返回0.这里的0实际上就是代表空指针,和NULL作用一样.\\








\end{document}
