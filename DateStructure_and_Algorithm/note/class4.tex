\documentclass[12pt, a4paper, oneside]{ctexart}
\usepackage{amsmath,extarrows, amsthm, amssymb, bm, graphicx, hyperref, geometry, mathrsfs,color}

\title{\huge\textbf{DS 第四课}}
\author{luojunxun}
\date{\today}
\linespread{2}%行间距
\geometry{left=2cm,right=2cm,top=2cm,bottom=2cm}%设置页面
\CTEXsetup[format={\Large\bfseries}]{section}%section左对齐

%定义环境
\newenvironment{Def}[1][def-name]{\par\noindent{\textit{(#1):}\small}}{\\\par}
\newenvironment{theorem}[1][Theorem-name]{\par\noindent \textbf{Theorem #1:}\textit}{\\\par}
\newenvironment{corollary}[1][corollary-name]{\par\noindent \textbf{Corollary #1:}\textit}{\\\par\vspace*{15pt}}
\newenvironment{lemma}[1][lemma-name]{\par\noindent \textbf{Lemma #1:}\textbf}{\\\par}
\renewenvironment{proof}{\par\noindent{\textit{Proof:}\small}}{\\\par}
\newenvironment{example}[1][example-name]{\par{\textbf{Example:}}}{\\\par}
\newenvironment{say}{\center{\textit{summary:}}}{\\\par}
\newenvironment{note}[1][note-name]{\par\textit{#1:}}{\\\par}
\newcommand{\qie}{\enspace\&\enspace}


\begin{document}
\maketitle

堆栈(stack,queue)

single linked list

push:插入一个新元素

pop:删除下一个元素

top(stack):读取顶部元素

front(queue):读取最前面的元素

size:求大小

empty:判断是否是空

stack:(LIFO)last in first out

queue:(FIFO)first in forst out































% \begin{figure}[p]

%     \centerline{\includegraphics[width=1.2\linewidth,height=1.1\textheight]{name}}
%     \caption{课上习题}
%     \label{figure}

%\end{figure}

i

% \bibliographystyle{IEEEtran}
% \bibliography{reference}



\end{document}
