\documentclass[12pt, a4paper, oneside]{ctexart}
\usepackage{amsmath,extarrows, amsthm, amssymb, bm, graphicx, hyperref, geometry, mathrsfs,color}

\title{\huge\textbf{DS homework03}}
\author{luojunxun}
\date{\today}
\linespread{2}%行间距
\geometry{left=2cm,right=2cm,top=2cm,bottom=2cm}%设置页面
\CTEXsetup[format={\Large\bfseries}]{section}%section左对齐

%定义环境
\newenvironment{Def}[1][def-name]{\par\noindent{\textit{(#1):}\small}}{\\\par}
\newenvironment{theorem}[1][Theorem-name]{\par\noindent \textbf{Theorem #1:}\textit}{\\\par}
\newenvironment{corollary}[1][corollary-name]{\par\noindent \textbf{Corollary #1:}\textit}{\\\par\vspace*{15pt}}
\newenvironment{lemma}[1][lemma-name]{\par\noindent \textbf{Lemma #1:}\textbf}{\\\par}
\renewenvironment{proof}{\par\noindent{\textit{Proof:}\small}}{\\\par}
\newenvironment{example}[1][example-name]{\par{\textbf{Example:}}}{\\\par}
\newenvironment{say}{\center{\textit{summary:}}}{\\\par}
\newenvironment{note}[1][note-name]{\par\textit{#1:}}{\\\par}
\newcommand{\qie}{\enspace\&\enspace}




\begin{document}

\title{项目报告:中缀表达式转后缀表达式计算器}
\author{罗俊勋}
\date{\today}
\maketitle

\section{项目介绍}
这个项目是一个中缀表达式转后缀表达式并计算表达式值的计算器。它包括了三个关键文件:'main.cpp','stringPrepoce.h',和'calculator.h'。下面是对项目的详细解释。

\section{设计思路}
在'main.cpp'文件中,我们首先读取命令行参数的中缀表达式,并通过'stringPreproce'函数进行预处理。预处理包括去除无关字符、检查括号和操作符的合法性。然后,使用'strToQueue'函数将预处理后的中缀表达式转换为后缀表达式队列,同时处理负数的情况。最后,使用'evalPostfix'函数计算后缀表达式的值,将结果输出到标准输出。

在'stringPrepoce.h'文件中,我们定义了一系列用于预处理中缀表达式的函数,包括错误提示函数、判断字符是否为操作符的函数、检查括号合法性的函数、检查操作符合法性的函数等。这些函数协同工作以确保中缀表达式的合法性。

在'calculator.h'文件中,我们定义了用于计算后缀表达式的函数。这包括判断字符串是否为数字的函数和执行后缀表达式计算的函数。'evalPostfix'函数遍历后缀表达式队列并使用双栈来执行计算。

\subsection{main.cpp}
'main.cpp'是项目的入口文件,负责接收用户输入的中缀表达式,处理它并输出计算结果。以下是'main.cpp'的关键设计思路:

\begin{itemize}
\item 引入必要的头文件:'iostream'、'stack'、'queue'、'string'等,以及自定义的'stringPreproce.h'和'calculator.h',以支持项目所需的功能。

\item 实现了op\_Priority函数,用于判断操作符的优先级,根据操作符的最后一个字符来确定其优先级。这是在处理操作符时非常重要的功能。

\item 实现了'countDots'函数,用于统计字符串中小数点的个数。这对于检测小数点的合法性很有帮助。

\item 主要逻辑包括将中缀表达式转化为后缀表达式,然后计算后缀表达式的值,并输出结果。

\item 处理了中缀表达式中的数字、操作符和括号。使用队列'infixQueue'来存储中缀表达式,栈'opStack'用于操作符的处理,队列'Postfix'用于存储后缀表达式。

\item 通过多次遍历中缀表达式,根据操作符的优先级将操作符转换为后缀表达式。最终,计算后缀表达式的值并输出。

\end{itemize}

\subsection{calculator.h}
'calculator.h'包含了计算后缀表达式的函数,以及一些辅助函数。以下是'calculator.h'的设计思路:

\begin{itemize}
\item 实现了'isnum'函数,用于判断一个字符串是否为数字,这是后缀表达式计算中的重要判断条件。

\item 实现了'evalPostfix'函数,该函数用于执行后缀表达式的计算。它使用一个操作数栈和遍历后缀表达式的方式来计算结果。

\item 实现了'eval'函数,用于执行二元运算,包括加法、减法、乘法和除法。根据操作符执行相应的运算,并返回结果。

\item 这些函数协同工作,将后缀表达式队列的计算结果返回给'main.cpp',并输出到标准输出。

\end{itemize}

\subsection{stringPrepoce.h}
'stringPrepoce.h'包含了一系列用于中缀表达式预处理的函数。以下是'stringPrepoce.h'的设计思路:

\begin{itemize}
\item 实现了错误提示函数err\_exp,用于在遇到非法表达式时输出错误信息并退出程序。

\item 实现了一系列函数,包括判断字符是否是操作符、四则运算符、小数点,以及检查括号合法性和操作符合法性的函数。

\item 定义了总判断函数strPreproce,它将中缀表达式进行预处理,去除无关字符,并检查括号和操作符的合法性。

\item 实现了'strToQueue'函数,用于将预处理后的中缀表达式字符串转换为队列,以便后续处理。它还处理了负数的情况,确保负号与数字正确解析。

\item 这些函数协同工作,确保中缀表达式的合法性,并将处理后的表达式传递给'main.cpp'。

\end{itemize}

\section{测试说明}

\subsection{单项测试}

\begin{itemize}

\item 如需测试输入”2+9*(9-7)“是否合法,在终端中输入 "./test "2+9*(9-7)" "即可

\item 如需大量测试,将测试用例按行输入input.txt后运行"bash run",即可在生成的output.txt文件中获得对应的输出

\end{itemize}

\end{document}
