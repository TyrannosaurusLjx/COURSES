\documentclass[12pt, a4paper, oneside]{ctexart}
\usepackage{amsmath,extarrows, amsthm, amssymb, bm, graphicx, hyperref, geometry, mathrsfs,color}

\title{\huge\textbf{概率论第一章节}}
\author{luojunxun}
\date{\today}
\linespread{2}%行间距
\geometry{left=2cm,right=2cm,top=2cm,bottom=2cm}%设置页面
\CTEXsetup[format={\Large\bfseries}]{section}%section左对齐

%定义环境
\newenvironment{Def}[1][def-name]{\par\noindent{\textit{(#1):}\small}}{\\\par}
\newenvironment{theorem}[1][Theorem-name]{\par\noindent \textbf{Theorem #1:}\textit}{\\\par}
\newenvironment{corollary}[1][corollary-name]{\par\noindent \textbf{Corollary #1:}\textit}{\\\par\vspace*{15pt}}
\newenvironment{lemma}[1][lemma-name]{\par\noindent \textbf{Lemma #1:}\textbf}{\\\par}
\renewenvironment{proof}{\par\noindent{\textit{Proof:}\small}}{\\\par}
\newenvironment{example}[1][example-name]{\par{\textbf{Example:}}}{\\\par}
\newenvironment{say}{\center{\textit{summary:}}}{\\\par}
\newenvironment{note}[1][note-name]{\par\textit{#1:}}{\\\par}
\newcommand{\qie}{\enspace\&\enspace}


\begin{document}
\maketitle

随机变量:$\xi(\omega)$是定义在概率空间$\{\Omega,\mathcal{F},P\}$上的单值实函数,对于R上任何一个波雷尔集合B,有
$\xi^{-1}(B)=\{\omega:\xi(\omega)\in B\}\in\mathcal{F}$,就称其为随机变量,$\{P(\xi(\omega))\in B\},B\in\mathcal{B}^{-1}$,称为随机变量的概率分布

常见分布:单点分布,两点分布(伯努利分布),二项分布$\xi ~ B(n,p)$且$P(\xi = k)=b(k;n,p)$

Possion定理:$if\;\exists\lambda>0,s.t.\lim\limits_{n\to\infty}np_n=\lambda,then\Rightarrow \lim\limits_{n\to\infty}b(k;n,p)=\frac{\lambda^k}{k!}e^{-\lambda} $
















% \begin{figure}[p]

%     \centerline{\includegraphics[width=1.2\linewidth,height=1.1\textheight]{name}}
%     \caption{课上习题}
%     \label{figure}

%\end{figure}



% \bibliographystyle{IEEEtran}
% \bibliography{reference}



\end{document}
