\documentclass[12pt, a4paper, oneside]{ctexart}
\usepackage{amsmath,extarrows, amsthm, amssymb, bm, graphicx, hyperref, geometry, mathrsfs,color}

\title{\huge\textbf{概率论第一章节}}
\author{luojunxun}
\date{\today}
\linespread{2}%行间距
\geometry{left=2cm,right=2cm,top=2cm,bottom=2cm}%设置页面
\CTEXsetup[format={\Large\bfseries}]{section}%section左对齐

%定义环境
\newenvironment{Def}[1][def-name]{\par\noindent{\textit{(#1):}\small}}{\\\par}
\newenvironment{theorem}[1][Theorem-name]{\par\noindent \textbf{Theorem #1:}\textit}{\\\par}
\newenvironment{corollary}[1][corollary-name]{\par\noindent \textbf{Corollary #1:}\textit}{\\\par\vspace*{15pt}}
\newenvironment{lemma}[1][lemma-name]{\par\noindent \textbf{Lemma #1:}\textbf}{\\\par}
\renewenvironment{proof}{\par\noindent{\textit{Proof:}\small}}{\\\par}
\newenvironment{example}[1][example-name]{\par{\textbf{Example:}}}{\\\par}
\newenvironment{say}{\center{\textit{summary:}}}{\\\par}
\newenvironment{note}[1][note-name]{\par\textit{#1:}}{\\\par}
\newcommand{\qie}{\enspace\&\enspace}


\begin{document}
\maketitle

事件域$\mathscr{F}$:全集在其中,关于余运算封闭,可列并封闭,可列交封闭

实际上,borel$\sigma$域取开区间,闭区间,左闭右开区间的时候是等价的

概率函数三个性质:非负性,规范性,可列可加性

$P(A) = 1-P(\bar{A}) $

若$B\subset A,P(A-B) =P(A)-P(B)  $

$P(A-B)=P(A)-P(AB)$

$P(A\bigcup B)=P(A) +P(B)-P(AB) $

多还少补公式$P(\bigcup_{i=1}^nA_i)=\sum\limits_{i=1}^nP(A_i) -\sum\limits_{1\leq i<j\leq n}P(A_iA_j)+\cdots+(-1)^{n-1}P(A_1A_2\cdots A_n) $

次可加性$P(\sum\limits_{i=1}^n)\leq \sum\limits_{i=1}^n P(A_i) $

如果一系列事件有极限,那么极限和概率函数可交换(满足交换律)

\section*{条件概率}

若$P(A_1\cdots A_{n-1})\neq 0,P(A_1A_2\cdots A_n)=P(A_1)P(A_2|A_1)P(A_3|A_1A_2)\cdots P(A_n|A_1\cdots A_{n-1}) $



















% \begin{figure}[p]

%     \centerline{\includegraphics[width=1.2\linewidth,height=1.1\textheight]{name}}
%     \caption{课上习题}
%     \label{figure}

%\end{figure}



% \bibliographystyle{IEEEtran}
% \bibliography{reference}



\end{document}
