\documentclass{ctexart}

\usepackage{graphicx}
\usepackage{amsmath}

\title{作业一: 洛必达法则的叙述与证明}


\author{罗俊勋 \\ 数学与应用数学 3210101613}

\begin{document}

\maketitle


    这是我们用来求待定型分式的极限的方法
\section{问题描述}
问题叙述如下: 设函数$f(x)$和$g(x)$在区间$(a,a+d]$上可导,(d是某个正常数),且$g'(x)\neq 0$,若此时有
  \[
  \lim\limits_{x \rightarrow a^{+}}g(x) =\lim\limits_{x\to a^{+}}f(x)=0
  \]
  或者有
  \[
  \lim\limits_{x \rightarrow a^{+} }g(x)=\infty
  \]
  且$\lim\limits_{x \rightarrow a^+}\frac{f'(x)}{g'(x)}$存在(可以是有限数或者$\infty$),则成立
  \[
\lim\limits_{x \rightarrow a^+}\frac{f(x)}{g(x)}=\lim\limits_{x \rightarrow a^+}\frac{f'(x)}{g'(x)}
  \]

  \section{证明}

  这里仅对$\lim\limits_{x \rightarrow a^+}\frac{f'(x)}{g'(x)}=A$为有限数时来证明,当$A$为无穷大时,证明过程是类似的,
  先证明$ \lim\limits_{x \rightarrow a^{+}}g(x) =\lim\limits_{x\to a^{+}}f(x)=0$的情况:
  \\$\quad$由于函数在$x=a$处的值与$x \rightarrow a^+$时的极限无关,因此可以补充定义
  \[
  f(a)=g(a)=0
  \]
  使得$f(x)$和$g(x)$在$[a,a+d]$上连续,这样,经补充定义后的函数$f(x)$和$g(x)$在$[a,a+d]$上满足$Cauchy$中值定理的条件,因此对于任意$x\in (a,a+d)$,存在$\xi \in (a,a+d)$,满足
  \[
  \frac{f(x)}{g(x)}=\frac{f(x)-f(a)}{g(x)-g(a)}=\frac{f'(\xi)}{g'(\xi)}
  \]
  当$x \rightarrow a^+$时显然有$\xi \rightarrow a^+$,由于$\lim\limits_{x \rightarrow a^+} \frac{f'(x)}{g'(x)}$存在,两端令$x \rightarrow a^+$,即有
  \[
  \lim\limits_{x \rightarrow a^+}\frac{f(x)}{g(x)}= \lim\limits_{x \rightarrow a^+}\frac{f\xi)}{g(\xi)}= \lim\limits_{x \rightarrow a^+}\frac{f'(x)}{g'(x)}
  \]

  下面证明$\lim\limits_{x \rightarrow a^+}g(x)=\infty$的情况
  记$x_0$是$(a,a+d]$中任意一个固定点,则当$x \neq x_0$,$\frac{f(x)}{g(x)}$可以改写为
  \[
 \frac{f(x)}{g(x)}=\frac{f(x)-f(x_0)}{g(x)}+\frac{f(x)}{g(x_0 }=\frac{g(x)-g(x_0)}{g(x)} \cdot \frac{f(x)-f(x_0)}{g(x)-g(x_0)}+\frac{f(x_0)}{g(x_0)}  
 \]
 \[
 =[1-\frac{g(x_0)}{g(x)}] \frac{f(x)-f(x_0)}{g(x)-g(x_0)}+\frac{f(x_0)}{g(x_0)}
 \]
 于是,
 \[
 |\frac{f(x)}{g(x)}-A|=|[1-\frac{g(x_0)}{g(x)}] \frac{f(x)-f(x_0)}{g(x)-g(x_0)}+\frac{f(x_0)}{g(x_0)}-A|
 \]
 \[
 \leq |[1-\frac{g(x_0)}{g(x)}]|\cdot| \frac{f(x)-f(x_0)}{g(x)-g(x_0)}-A|+|\frac{f(x_0)-Ag(x_0)}{g(x)}|
 \]
 因为$\lim\limits_{x \to a^+}\frac{f'(x)}{g'(x)}=A$,所以对于任意$\epsilon \textgreater 0$,存在$\rho \textgreater 0$,当$0\textless x-a \textless \rho$时
 \[
 |\frac{f'(x)}{g'(x)}-A|\textless \epsilon
 \]
 取$x_0=a+\rho$,由$Cauchy$中值定理,对于任意$x \in (a,x_0)$,存在$\xi\in(x,x_0)\subset(a,a+\rho) $满足
 \[
 \frac{f(x)-f(x_0)}{g(x)-g(x_0)}=\frac{f'(\xi)}{g'(\xi)}
 \]
 于是得到
 \[
| \frac{f(x)-f(x_0)}{g(x)-g(x_0)}-A|=|\frac{f'(\xi)}{g'(\xi)}-A|\textless \epsilon
 \]

 又因为$\lim\limits_{x \to a^+}g(x)=\infty$,所以可以找到正数$\delta \textless \rho$,当$0\textless x-a\textless \delta$时,成立
 \[
|[1-\frac{g(x_0)}{g(x)}]|\textless 2,+\frac{f(x_0)-Ag(x_0)}{g(x)}|\textless \epsilon
 \]

 综上所述,即知对于任意的$\epsilon \textgreater 0$,存在$\delta \textgreater 0$,当$0 \textless x-a \textless \delta$时
 \[
 |\frac{f(x)}{g(x)}-A|\textless |1-\frac{g(x_0)}{g(x)}| \cdot | \frac{f(x)-f(x_0)}{g(x)-g(x_0)}-A|+|\frac{f(x_0)-Ag(x_0)}{g(x)}|\textless 2\epsilon+\epsilon=3\epsilon
 \]

 由定义即得
 \[
\lim\limits_{x \to a^+}\frac{f(x)}{g(x}=A=\lim\limits_{x \to a^+}\frac{f'(x)}{g'(x)}
 \]
   证毕!

 


\end{document}
