\documentclass[12pt, a4paper, oneside]{ctexart}
\usepackage{amsmath,extarrows, amsthm, amssymb, bm, graphicx, hyperref, geometry, mathrsfs,color}

\title{\huge\textbf{计算机模拟第二课}}
\author{luojunxun}
\date{\today}
\linespread{2}%行间距
\geometry{left=2cm,right=2cm,top=2cm,bottom=2cm}%设置页面
\CTEXsetup[format={\Large\bfseries}]{section}%section左对齐

%定义环境
\newenvironment{Def}[1][def-name]{\par\noindent{\textit{(#1):}\small}}{\\\par}
\newenvironment{theorem}[1][Theorem-name]{\par\noindent \textbf{Theorem #1:}\textit}{\\\par}
\newenvironment{corollary}[1][corollary-name]{\par\noindent \textbf{Corollary #1:}\textit}{\\\par\vspace*{15pt}}
\newenvironment{lemma}[1][lemma-name]{\par\noindent \textbf{Lemma #1:}\textbf}{\\\par}
\renewenvironment{proof}{\par\noindent{\textit{Proof:}\small}}{\\\par}
\newenvironment{example}[1][example-name]{\par{\textbf{Example:}}}{\\\par}
\newenvironment{say}{\center{\textit{summary:}}}{\\\par}
\newenvironment{note}[1][note-name]{\par\textit{#1:}}{\\\par}
\newcommand{\qie}{\enspace\&\enspace}


\begin{document}
\maketitle
随机数生成器:1.平方取中法;2.经典fibonacci生成器;3.线性同余

1.平方取中法:取一个2s位的整数,称为种子,将其平凡得到一个4s位的整数,$x_{i+1}=[\frac{x_i^2}{10^s}]mod10^{2s}$
取这4s位中间2s位作为下一个种子数,并归一化运算得到[0,1]之间的数$u_{i+1}=\frac{x_{i+1}}{10^{2s}}$

1.1乘法取中法,常数乘子法等等

2.经典fibonacci生成器 $x_i=(x_{i-1}+x_{i-1})\;mod\; M$产生[0,M-1]之间的随机数

3.线性同余法:$x_i=(a*x_{i-1}+c)\;mod\;m$

4.非线性同余器:

5.梅森旋转


随机数检验:1.理论检验;2.统计检验(一般检验按十进制检查,严格检验则为二进制)




















% \begin{figure}[p]

%     \centerline{\includegraphics[width=1.2\linewidth,height=1.1\textheight]{name}}
%     \caption{课上习题}
%     \label{figure}

%\end{figure}



% \bibliographystyle{IEEEtran}
% \bibliography{reference}



\end{document}
