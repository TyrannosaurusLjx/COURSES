\documentclass[12pt, a4paper, oneside]{ctexart}
\usepackage{amsmath,extarrows, amsthm, amssymb, bm, graphicx, hyperref, geometry, mathrsfs,color}

\title{\huge\textbf{集合与实数集}}
\author{luojunxun}
\date{\today}
\linespread{2}%行间距
\geometry{left=2cm,right=2cm,top=2cm,bottom=2cm}%设置页面
\CTEXsetup[format={\Large\bfseries}]{section}%section左对齐

%定义环境
\newenvironment{Def}[1][def-name]{\par\noindent{\textit{(#1):}\small}}{\\\par}
\newenvironment{theorem}[1][Theorem-name]{\par\noindent \textbf{Theorem #1:}\textit}{\\\par}
\newenvironment{corollary}[1][corollary-name]{\par\noindent \textbf{Corollary #1:}\textit}{\\\par\vspace*{15pt}}
\newenvironment{lemma}[1][lemma-name]{\par\noindent \textbf{Lemma #1:}\textbf}{\\\par}
\renewenvironment{proof}{\par\noindent{\textit{Proof:}\small}}{\\\par}
\newenvironment{example}[1][example-name]{\par{\textbf{Example:}}}{\\\par}
\newenvironment{say}{\center{\textit{summary:}}}{\\\par}
\newenvironment{note}[1][note-name]{\par\textit{#1:}}{\\\par}
\newcommand{\qie}{\quad\&\quad}


\begin{document}
\maketitle


要证明$m(E) = 1$,可以采用反证法。假设$m(E) < 1$,那么存在一个开区间$(a,b) \subset [0,1]$,使得$m(E \cap (a,b)) < b - a$。

由于$E \subset [0,1]$,所以$E \cap (a,b) \subset (a,b)$。因此,对于任意的$I = (c,d) \subset (a,b)$,有:

�
(
�
∩
�
)
≤
�
(
�
∩
(
�
,
�
)
)
<
�
−
�
m(E∩I)≤m(E∩(a,b))<b−a

又因为$E$是可测集,所以$E \cap (a,b)$也是可测集。根据可测集的定义,对于任意的$\epsilon > 0$,都存在开集$O \supset E \cap (a,b)$,使得$m(O \setminus (E \cap (a,b))) < \epsilon$。因为开集是可数个开区间的并集,所以存在一个开区间$I \subset (a,b)$,使得$I \cap (O \setminus (E \cap (a,b))) \neq \emptyset$。

因此,$I \cap O \subset (a,b)$,且$I \cap O \subset E^c$。于是,$m(E \cap I) \leq m(O \setminus (E \cap (a,b))) < \epsilon$。取$\epsilon = \frac{b-a}{2}$,得到$m(E \cap I) < \frac{b-a}{2}$。这与$m(E \cap I) \leq \delta(b-a)$矛盾,因为$\delta < \frac{1}{2}$。因此,假设$m(E) < 1$不成立,即$m(E) = 1$。

















% \begin{figure}[p]

%     \centerline{\includegraphics[width=1.2\linewidth,height=1.1\textheight]{name}}
%     \caption{课上习题}
%     \label{figure}

%\end{figure}



% \bibliographystyle{IEEEtran}
% \bibliography{reference}



\end{document}